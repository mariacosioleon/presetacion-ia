\documentclass{article}
\usepackage{fleqn}
\usepackage{epsf}
\usepackage{aima2e-slides}

\begin{document}

\begin{huge}
\titleslide{Uncertainty}{Chapter 13}

\sf

%%%%%%%%%%%% Slide %%%%%%%%%%%%%%%%%%%%%%%%%%%%%%%%%%%%%%%%%%%%%%%%%%%
\heading{Outline}

\blob Uncertainty

\blob Probability

\blob Syntax and Semantics

\blob Inference

\blob Independence and Bayes' Rule


%%%%%%%%%%%% Slide %%%%%%%%%%%%%%%%%%%%%%%%%%%%%%%%%%%%%%%%%%%%%%%%%%%
\heading{Uncertainty}

Let action \mat{$A_t$} = leave for airport \mat{$t$} minutes before flight\\
Will \mat{$A_t$} get me there on time?

Problems:\al
1) partial observability (road state, other drivers' plans, etc.)\al
2) noisy sensors (KCBS traffic reports)\al
3) uncertainty in action outcomes (flat tire, etc.)\al
4) immense complexity of modelling and predicting traffic

Hence a purely logical approach either\\
\phantom{or }1) risks falsehood: ``\mat{$A_{25}$} will get me there on time''\\
or 2) leads to conclusions that are too weak for decision making:\nl
``\mat{$A_{25}$} will get me there on time if there's no accident on the bridge\nl
and it doesn't rain and my tires remain intact etc etc.''

(\mat{$A_{1440}$} might reasonably be said to get me there on time\\
but I'd have to stay overnight in the airport $\ldots$)



%%%%%%%%%%%% Slide %%%%%%%%%%%%%%%%%%%%%%%%%%%%%%%%%%%%%%%%%%%%%%%%%%%
\heading{Methods for handling uncertainty}

\defn{Default} or \defn{nonmonotonic} logic:\al
  Assume my car does not have a flat tire\al
  Assume \mat{$A_{25}$} works unless contradicted by evidence\\
Issues: What assumptions are reasonable? How to handle contradiction?

\defn{Rules with fudge factors}:\al
\mat{$A_{25} \mapsto_{0.3} AtAirportOnTime$}\al
\mat{$Sprinkler \mapsto_{0.99} WetGrass$}\al
\mat{$WetGrass \mapsto_{0.7} Rain$}\\
Issues: Problems with combination, e.g., \mat{$Sprinkler$} causes \mat{$Rain$}??

\defn{Probability}\al
  Given the available evidence,\nl
    \mat{$A_{25}$} will get me there on time with probability \mat{$0.04$}\\
Mahaviracarya (9th C.), Cardamo (1565) theory of gambling

(\defn{Fuzzy logic} handles \emph{degree of truth} NOT uncertainty e.g.,\al
  \mat{$WetGrass$} is true to degree \mat{$0.2$})



%%%%%%%%%%%% Slide %%%%%%%%%%%%%%%%%%%%%%%%%%%%%%%%%%%%%%%%%%%%%%%%%%%
\heading{Probability}

Probabilistic assertions \emph{summarize} effects of\al
  \note{laziness}: failure to enumerate exceptions, qualifications, etc.\al
  \note{ignorance}: lack of relevant facts, initial conditions, etc.

\defn{Subjective} or \defn{Bayesian} probability:\\
Probabilities relate propositions to one's own state of knowledge\nl
e.g., \mat{$P(A_{25} | \mbox{no reported accidents}) = 0.06$}

These are \emph{not} claims of a ``probabilistic
tendency'' in the current situation\\
(but might be learned from past experience of similar situations)

Probabilities of propositions change with new evidence:\nl
e.g., \mat{$P(A_{25} | \mbox{no reported accidents},\ \mbox{5 a.m.}) = 0.15$}

(Analogous to logical entailment status \mat{$KB \models \alpha$}, not truth.)



%%%%%%%%%%%% Slide %%%%%%%%%%%%%%%%%%%%%%%%%%%%%%%%%%%%%%%%%%%%%%%%%%%
\heading{Making decisions under uncertainty}

Suppose I believe the following:
\mat{\begin{eqnarray*}
P(A_{25}\mbox{ gets me there on time} | \ldots) &=& 0.04 \\
P(A_{90}\mbox{ gets me there on time} | \ldots) &=& 0.70 \\
P(A_{120}\mbox{ gets me there on time} | \ldots) &=& 0.95 \\
P(A_{1440}\mbox{ gets me there on time} | \ldots) &=& 0.9999 
\end{eqnarray*}}
Which action to choose?

Depends on my \defn{preferences} for missing flight vs. airport cuisine, etc.

\defn{Utility theory} is used to represent and infer preferences

\defn{Decision theory} = utility theory + probability theory



%%%%%%%%%%%% Slide %%%%%%%%%%%%%%%%%%%%%%%%%%%%%%%%%%%%%%%%%%%%%%%%%%%
\heading{Probability basics}

Begin with a set \mat{$\Omega$}---the \defn{sample space}\al
e.g., 6 possible rolls of a die.\al
\mat{$\omega\in \Omega$} is a \defn{sample point}/\defn{possible world}/\defn{atomic event}

A \defn{probability space} or \defn{probability model} is a sample space\\
with an assignment \mat{$P(\omega)$} for every \mat{$\omega\in\Omega$} s.t.\al
  \mat{$0\leq P(\omega) \leq 1$}\al
  \mat{$\mysum_{\omega} P(\omega) = 1$}\\
e.g., \mat{$P(1)\eq P(2)\eq P(3)\eq P(4)\eq P(5)\eq P(6) \eq 1/6$}.

An \defn{event} \mat{$A$} is any subset of \mat{$\Omega$}
\mat{\[
  P(A) = \mysum_{\{\omega\in A\}} P(\omega)
\]}
E.g., \mat{$P(\mbox{die roll}<4) = P(1) + P(2) + P(3) = 1/6 + 1/6 + 1/6 = 1/2$}


%%%%%%%%%%%% Slide %%%%%%%%%%%%%%%%%%%%%%%%%%%%%%%%%%%%%%%%%%%%%%%%%%%
\heading{Random variables}

A \defn{random variable} is a function from sample points
to some range, e.g., the reals or Booleans\al
e.g., \mat{$Odd(1)\eq true$}.

\mat{$P$} induces a \defn{probability distribution} for any r.v. \mat{$X$}:
\mat{\[
  P(X\eq x_i) = \mysum_{\{\omega: X(\omega)\eq x_i\}} P(\omega)
\]}
e.g., \mat{$P(Odd\eq true) = P(1) + P(3) + P(5) = 1/6 + 1/6 + 1/6 = 1/2$}

%%%%%%%%%%%% Slide %%%%%%%%%%%%%%%%%%%%%%%%%%%%%%%%%%%%%%%%%%%%%%%%%%%
\heading{Propositions}

Think of a proposition as the event (set of sample points)\\
where the proposition is true

Given Boolean random variables \mat{$A$} and \mat{$B$}:\al
  event \mat{$a$} = set of sample points where \mat{$A(\omega)\eq true$}\al
  event \mat{$\lnot a$} = set of sample points where \mat{$A(\omega)\eq false$}\al
  event \mat{$a\land b$} = points where \mat{$A(\omega)\eq true$} and \mat{$B(\omega)\eq true$}

Often in AI applications, the sample points are \emph{defined}\\
by the values of a set of random variables, i.e., the\\
sample space is the Cartesian product of the ranges of the variables

With Boolean variables, sample point = propositional logic model\nl
  e.g., \mat{$A\eq true$}, \mat{$B\eq false$}, or \mat{$a\land \lnot b$}.\\
Proposition = disjunction of atomic events in which it is true\al
e.g., \mat{$(a\lor b) \equiv (\lnot a \land b) \lor (a \land \lnot b) \lor (a \land b) $}\al
\mat{$\implies P(a\lor b) = P(\lnot a \land b) + P(a \land \lnot b) + P(a \land b) $}


%%%%%%%%%%%% Slide %%%%%%%%%%%%%%%%%%%%%%%%%%%%%%%%%%%%%%%%%%%%%%%%%%%
\heading{Why use probability?}

The definitions imply that certain logically related events
must have related probabilities

E.g., \mat{$P(a \lor b) = P(a) + P(b) - P(a\land b)$}

\vspace*{0.2in}

\epsfxsize=0.45\textwidth
\fig{\file{figures}{axiom3-venn.ps}}

de Finetti (1931): an agent who bets according to probabilities that violate
these axioms can be forced to bet so as to lose money regardless of outcome.

%%%%%%%%%%%% Slide %%%%%%%%%%%%%%%%%%%%%%%%%%%%%%%%%%%%%%%%%%%%%%%%%%%
\heading{Syntax for propositions}

\defn{Propositional} or \defn{Boolean} random variables\al
  e.g., \mat{$Cavity$} (do I have a cavity?)\al
  \mat{$Cavity\eq true$} is a proposition, also written \mat{$cavity$}

\defn{Discrete} random variables (\note{finite} or \note{infinite}) \al
  e.g., \mat{$Weather$} is one of \mat{$\<sunny,rain,cloudy,snow\>$}\al
  \mat{$Weather\eq rain$} is a proposition\al
Values must be exhaustive and mutually exclusive

\defn{Continuous} random variables (\note{bounded} or \note{unbounded})\al
  e.g., \mat{$Temp\eq 21.6$}; also allow, e.g., \mat{$Temp<22.0$}.

Arbitrary Boolean combinations of basic propositions

%%%%%%%%%%%% Slide %%%%%%%%%%%%%%%%%%%%%%%%%%%%%%%%%%%%%%%%%%%%%%%%%%%
\heading{Prior probability}

\defn{Prior} or \defn{unconditional probabilities} of propositions\al
  e.g., \mat{$P(Cavity\eq true) = 0.1$} and \mat{$P(Weather \eq sunny) = 0.72$}\\
correspond to belief prior to arrival of any (new) evidence

\defn{Probability distribution} gives values for all possible assignments:\al
  \mat{$\pv(Weather) = \<0.72,0.1,0.08,0.1\>$} (\note{normalized}, i.e., 
       sums to \mat{$1$})

\defn{Joint probability distribution} for a set of r.v.s gives the\\
probability of every atomic event on those r.v.s (i.e., every sample point)\al
  \mat{$\pv(Weather,Cavity)$} = a \mat{$4 \times 2$} matrix of values:\\
\mat{\[\begin{array}{l|llll}
\hfil Weather \eq & sunny & rain & cloudy & snow \\
\hline
Cavity \eq true &0.144 &0.02 &0.016 & 0.02\\
Cavity \eq false&0.576 &0.08 &0.064 & 0.08
\end{array}\]}
\emph{Every question about a domain can be answered by the joint\\
distribution because every event is a sum of sample points}

%%%%%%%%%%%% Slide %%%%%%%%%%%%%%%%%%%%%%%%%%%%%%%%%%%%%%%%%%%%%%%%%%%
\heading{Probability for continuous variables}

Express distribution as a parameterized function of value:\al
\mat{$P(X\eq x) = U[18,26](x)$} = uniform density between \mat{$18$} and \mat{$26$}

\vspace*{0.2in}

\epsfxsize=0.65\textwidth
\fig{\file{figures}{uniform-density.ps}}

Here \mat{$P$} is a \defn{density}; integrates to 1.\\
\mat{$P(X\eq 20.5)=0.125$} really means
\mat{\[
  \lim_{dx \to 0} P(20.5 \leq X \leq 20.5+dx)/dx = 0.125
\]}

%%%%%%%%%%%% Slide %%%%%%%%%%%%%%%%%%%%%%%%%%%%%%%%%%%%%%%%%%%%%%%%%%%
\heading{Gaussian density}

\mat{$P(x) = \frac{1}{\sqrt{2\pi} \sigma} e^{-(x-\mu)^2/2\sigma^2}$}

\vspace*{0.2in}

\epsfxsize=0.65\textwidth
\fig{\file{figures}{gaussian-density.ps}}

%%%%%%%%%%%% Slide %%%%%%%%%%%%%%%%%%%%%%%%%%%%%%%%%%%%%%%%%%%%%%%%%%%
\heading{Conditional probability}

\defn{Conditional} or \defn{posterior probabilities}\al
  e.g., \mat{$P(cavity | toothache) = 0.8$}\al
  i.e., \emph{given that} \mat{$toothache$} \emph{is all I know}\al
  \emph{NOT} ``if \mat{$toothache$} then 80\% chance of \mat{$cavity$}''

(Notation for conditional distributions:\al
  \mat{$\pv(Cavity |Toothache)$} = 2-element vector of 2-element vectors)

If we know more, e.g., \mat{$cavity$} is also given, then we have\al
  \mat{$P(cavity | toothache,cavity) = 1$}\\
Note: the less specific belief \emph{remains valid} after more evidence 
arrives, but is not always \emph{useful}

New evidence may be irrelevant, allowing simplification, e.g.,\al
\mat{$P(cavity | toothache,49ersWin) = P(cavity | toothache) = 0.8$}\\
This kind of inference, sanctioned by domain knowledge, is crucial

%%%%%%%%%%%% Slide %%%%%%%%%%%%%%%%%%%%%%%%%%%%%%%%%%%%%%%%%%%%%%%%%%%
\heading{Conditional probability}

Definition of conditional probability:
\mat{\[
  P(a|b) = \frac{P(a\land b)}{P(b)} \mbox{ if } P(b) \neq 0
\]}
\defn{Product rule} gives an alternative formulation:\al
  \mat{$P(a\land b) = P(a|b)P(b) = P(b|a)P(a)$}

A general version holds for whole distributions, e.g.,\al
  \mat{$\pv(Weather,Cavity) = \pv(Weather|Cavity) \pv(Cavity)$}\\
(View as a \mat{$4\times 2$} set of equations, \emph{not} matrix mult.)

\defn{Chain rule} is derived by successive application of product rule:\al
\mat{$\pv(X_1,\ldots,X_n) = \pv(X_1,\ldots,X_{n-1})\ 
                        \pv(X_n | X_1,\ldots,X_{n-1})$}\nl
                    = \mat{$\pv(X_1,\ldots,X_{n-2})\ 
                        \pv(X_{n-1} | X_1,\ldots,X_{n-2})\ 
                        \pv(X_n | X_1,\ldots,X_{n-1})$}\nl
                  = \mat{$\ldots$}\nl
                  = \mat{$\myprod_{i\eq 1}^n \pv(X_i | X_1,\ldots,X_{i-1})$}


 

%%%%%%%%%%%% Slide %%%%%%%%%%%%%%%%%%%%%%%%%%%%%%%%%%%%%%%%%%%%%%%%%%%
\heading{Inference by enumeration}

Start with the joint distribution:

\epsfxsize=0.65\textwidth
\fig{\file{figures}{dentist-joint.ps}}

For any proposition \mat{$\phi$}, sum the atomic events where it is true:\al
 \mat{$P(\phi) = \mysum_{\omega: \omega \models \phi} P(\omega)$}

%%%%%%%%%%%% Slide %%%%%%%%%%%%%%%%%%%%%%%%%%%%%%%%%%%%%%%%%%%%%%%%%%%
\heading{Inference by enumeration}

Start with the joint distribution:

\epsfxsize=0.65\textwidth
\fig{\file{figures}{dentist-joint1.ps}}

For any proposition \mat{$\phi$}, sum the atomic events where it is true:\al
 \mat{$P(\phi) = \mysum_{\omega: \omega \models \phi} P(\omega)$}

\mat{$P(toothache) = 0.108 + 0.012 + 0.016 + 0.064 = 0.2$}

%%%%%%%%%%%% Slide %%%%%%%%%%%%%%%%%%%%%%%%%%%%%%%%%%%%%%%%%%%%%%%%%%%
\heading{Inference by enumeration}

Start with the joint distribution:

\epsfxsize=0.65\textwidth
\fig{\file{figures}{dentist-joint2.ps}}

For any proposition \mat{$\phi$}, sum the atomic events where it is true:\al
 \mat{$P(\phi) = \mysum_{\omega: \omega \models \phi} P(\omega)$}

\mat{$P(cavity \lor toothache) = 0.108 + 0.012 + 0.072 + 0.008 + 0.016 + 0.064 
                         = 0.28$}

%%%%%%%%%%%% Slide %%%%%%%%%%%%%%%%%%%%%%%%%%%%%%%%%%%%%%%%%%%%%%%%%%%
\heading{Inference by enumeration}

Start with the joint distribution:

\epsfxsize=0.65\textwidth
\fig{\file{figures}{dentist-joint3.ps}}

Can also compute conditional probabilities:
\mat{\begin{eqnarray*}
P(\lnot cavity|toothache) 
     &=& \frac{P(\lnot cavity \land toothache)}{P(toothache)}\\
     &=& \frac{0.016+0.064}{0.108 + 0.012 + 0.016 + 0.064} = 0.4
\end{eqnarray*}}

%%%%%%%%%%%% Slide %%%%%%%%%%%%%%%%%%%%%%%%%%%%%%%%%%%%%%%%%%%%%%%%%%%
\heading{Normalization}

\epsfxsize=0.65\textwidth
\fig{\file{figures}{dentist-joint4.ps}}

Denominator can be viewed as a \note{normalization constant} \mat{$\alpha$}
\mat{\begin{eqnarray*}
 \lefteqn{\pv(Cavity|toothache) = \alpha\, \pv(Cavity,toothache)} \\
  &=& \alpha\, [\pv(Cavity,toothache,catch)+\pv(Cavity,toothache,\lnot catch)]\\
  &=& \alpha\, [\<0.108,0.016\> + \<0.012,0.064\>] \\
  &=& \alpha\, \<0.12,0.08\> = \<0.6,0.4\>
\end{eqnarray*}}
General idea: compute distribution on query variable\\
by fixing \defn{evidence variables} and summing over \defn{hidden variables}

%%%%%%%%%%%% Slide %%%%%%%%%%%%%%%%%%%%%%%%%%%%%%%%%%%%%%%%%%%%%%%%%%%
\heading{Inference by enumeration, contd.}

Let \mat{$\mbf{X}$} be all the variables. Typically, we want \al
  the posterior joint distribution of the \defn{query variables} \mat{$\mbf{Y}$}\al
  given specific values \mat{$\mbf{e}$} for the \defn{evidence variables} \mat{$\mbf{E}$}

Let the \defn{hidden variables} be \mat{$\mbf{H} = \mbf{X} - \mbf{Y} - \mbf{E}$}

Then the required summation of joint entries is done by \note{summing out}
the hidden variables:
\mat{\[
\pv(\mbf{Y}|\mbf{E}\eq \mbf{e}) = \alpha \pv(\mbf{Y},\mbf{E}\eq \mbf{e})
= \alpha \mysum_{\smbf{h}} \pv(\mbf{Y},\mbf{E}\eq \mbf{e},\mbf{H}\eq \mbf{h})
\]}
The terms in the summation are joint entries because \mat{$\mbf{Y}$}, \mat{$\mbf{E}$}, and \mat{$\mbf{H}$} together exhaust the set of random variables

Obvious problems:\al
1) Worst-case time complexity \mat{$O(d^n)$} where \mat{$d$} is the largest arity\al
2) Space complexity \mat{$O(d^n)$} to store the joint distribution\al
3) How to find the numbers for \mat{$O(d^n)$} entries???


%%%%%%%%%%%% Slide %%%%%%%%%%%%%%%%%%%%%%%%%%%%%%%%%%%%%%%%%%%%%%%%%%%
\heading{Independence}

\mat{$A$} and \mat{$B$} are \defn{independent} iff\\
\mat{$\pv(A|B) \eq \pv(A)$}\quad or\quad \mat{$\pv(B|A) \eq \pv(B)$}\quad or\quad \mat{$\pv(A, B) \eq \pv(A)\pv(B)$}

\epsfxsize=0.75\textwidth
\fig{\file{figures}{weather-independence.ps}}

\mat{$\pv(Toothache,Catch,Cavity,Weather)$}\nl
\mat{${} = \pv(Toothache,Catch,Cavity)\pv(Weather)$}

32 entries reduced to 12; for \mat{$n$} independent biased coins, \mat{$2^n \rightarrow n$}

Absolute independence powerful but rare

Dentistry is a large field with hundreds of variables,\\
none of which are independent. What to do?

%%%%%%%%%%%% Slide %%%%%%%%%%%%%%%%%%%%%%%%%%%%%%%%%%%%%%%%%%%%%%%%%%%
\heading{Conditional independence}

\mat{$\pv(Toothache,Cavity,Catch)$} has \mat{$2^3 - 1$} = 7 independent entries

If I have a cavity, the probability that the probe catches in it
doesn't depend on whether I have a toothache:\al
  (1) \mat{$P(catch|toothache,cavity) = P(catch|cavity)$}

The same independence holds if I haven't got a cavity:\al
  (2) \mat{$P(catch|toothache,\lnot cavity) = P(catch|\lnot cavity)$}

\mat{$Catch$} is \defn{conditionally independent} of \mat{$Toothache$} given \mat{$Cavity$}:\al
  \mat{$\pv(Catch|Toothache,Cavity) = \pv(Catch|Cavity)$}

Equivalent statements:\al
   \mat{$\pv(Toothache|Catch,Cavity) = \pv(Toothache|Cavity)$}\al
   \mat{$\pv(Toothache,Catch|Cavity) = \pv(Toothache|Cavity)\pv(Catch|Cavity)$}

%%%%%%%%%%%% Slide %%%%%%%%%%%%%%%%%%%%%%%%%%%%%%%%%%%%%%%%%%%%%%%%%%%
\heading{Conditional independence contd.}

Write out full joint distribution using chain rule:\al
  \mat{$\pv(Toothache,Catch,Cavity)$}\al
  \mat{${} = \pv(Toothache|Catch,Cavity) \pv(Catch,Cavity)$}\al
  \mat{${} = \pv(Toothache|Catch,Cavity) \pv(Catch|Cavity)\pv(Cavity)$}\al
  \mat{${} = \pv(Toothache|Cavity) \pv(Catch|Cavity)\pv(Cavity)$}

I.e., 2 + 2 + 1 = 5 independent numbers (equations 1 and 2 remove 2)

\note{In most cases, the use of conditional independence reduces the size of
the representation of the joint distribution from exponential in \mat{$n$}
to linear in \mat{$n$}.}

\emph{Conditional independence is our most basic and robust \\
form of knowledge about uncertain environments.}

%%%%%%%%%%%% Slide %%%%%%%%%%%%%%%%%%%%%%%%%%%%%%%%%%%%%%%%%%%%%%%%%%%
\heading{Bayes' Rule}

Product rule \mat{$P(a\land b) = P(a|b)P(b) = P(b|a)P(a)$}
\mat{\[
{}\implies \mbox{\defn{Bayes' rule }}  P(a|b) = \frac{P(b|a)P(a)}{P(b)}
\]}
or in distribution form 
\mat{\[
\pv(Y|X) = \frac{\pv(X|Y)\pv(Y)}{\pv(X)} = \alpha \pv(X|Y)\pv(Y)
\]}
Useful for assessing \defn{diagnostic} probability from \defn{causal} probability:
\mat{\[
  P(Cause|Effect) = \frac{P(Effect|Cause)P(Cause)}{P(Effect)}
\]}
E.g., let \mat{$M$} be meningitis, \mat{$S$} be stiff neck:
\mat{\[
  P(m|s) = \frac{P(s|m)P(m)}{P(s)} = \frac{0.8 \times 0.0001}{0.1} = 0.0008
\]}
Note: posterior probability of meningitis still very small!

%%%%%%%%%%%% Slide %%%%%%%%%%%%%%%%%%%%%%%%%%%%%%%%%%%%%%%%%%%%%%%%%%%
\heading{Bayes' Rule and conditional independence}


\mat{\begin{eqnarray*}
\lefteqn{\pv(Cavity|toothache \land catch)}\\
   & = & \alpha\, \pv(toothache\land catch|Cavity) \pv(Cavity) \\
   & = & \alpha\, \pv(toothache|Cavity)\pv(catch|Cavity) \pv(Cavity) 
\end{eqnarray*}}
This is an example of a \defn{naive Bayes} model:
\mat{\[
  \pv(Cause,Effect_1,\ldots,Effect_n) =
       \pv(Cause)\myprod_i \pv(Effect_i|Cause)
\]}

\vspace*{0.3in}

\epsfxsize=0.75\textwidth
\fig{\file{figures}{naive-bayes.ps}}

Total number of parameters is \emph{linear} in \mat{$n$}

%%%%%%%%%%%% Slide %%%%%%%%%%%%%%%%%%%%%%%%%%%%%%%%%%%%%%%%%%%%%%%%%%%
\heading{Wumpus World}

\epsfxsize=0.35\textwidth
\fig{\file{figures}{wumpus-stuck.ps}}

\mat{$P_{ij}\eq true$} iff [\mat{$i,j$}] contains a pit

\mat{$B_{ij}\eq true$} iff [\mat{$i,j$}] is breezy\\
Include only \mat{$B_{1,1},B_{1,2},B_{2,1}$} in the probability model

%%%%%%%%%%%% Slide %%%%%%%%%%%%%%%%%%%%%%%%%%%%%%%%%%%%%%%%%%%%%%%%%%%
\heading{Specifying the probability model}

The full joint distribution is \mat{$\pv(P_{1,1},\ldots,P_{4,4},B_{1,1},B_{1,2},B_{2,1})$}

Apply product rule: \mat{$\pv(B_{1,1},B_{1,2},B_{2,1}\,|\,P_{1,1},\ldots,P_{4,4}) \pv(P_{1,1},\ldots,P_{4,4})$}

(Do it this way to get \mat{$P(Effect|Cause)$}.)

First term: 1 if pits are adjacent to breezes, 0 otherwise

Second term: pits are placed randomly, probability 0.2 per square:
\mat{\[
   \pv(P_{1,1},\ldots,P_{4,4}) = \myprod_{i,j\eq 1,1}^{4,4} \pv(P_{i,j}) = 0.2^n \stimes 0.8^{16-n}
\]}
for \mat{$n$} pits.

%%%%%%%%%%%% Slide %%%%%%%%%%%%%%%%%%%%%%%%%%%%%%%%%%%%%%%%%%%%%%%%%%%
\heading{Observations and query}

We know the following facts:\al
\mat{$b = \lnot b_{1,1} \land b_{1,2} \land b_{2,1}$}\al
\mat{$known = \lnot p_{1,1} \land \lnot p_{1,2}\land \lnot p_{2,1}$}

Query is \mat{$\pv(P_{1,3}|known,b)$}

Define \mat{$Unknown$} = \mat{$P_{ij}$}s other than \mat{$P_{1,3}$} and \mat{$Known$}

For inference by enumeration, we have
\mat{\[
  \pv(P_{1,3}|known,b) = \alpha \mysum_{unknown}\pv(P_{1,3},unknown,known,b)  
\]}
Grows exponentially with number of squares!

%%%%%%%%%%%% Slide %%%%%%%%%%%%%%%%%%%%%%%%%%%%%%%%%%%%%%%%%%%%%%%%%%%
\heading{Using conditional independence}

Basic insight: observations are conditionally independent of
other hidden squares given neighbouring hidden squares

\vspace*{0.2in}

\epsfxsize=0.35\textwidth
\fig{\file{figures}{wumpus-variables.ps}}

Define \mat{$Unknown = Fringe \cup Other$}\\
\mat{$\pv(b|P_{1,3},Known,Unknown) = \pv(b|P_{1,3},Known,Fringe)$}

Manipulate query into a form where we can use this!

%%%%%%%%%%%% Slide %%%%%%%%%%%%%%%%%%%%%%%%%%%%%%%%%%%%%%%%%%%%%%%%%%%
\heading{Using conditional independence contd.}

\mat{\begin{eqnarray*}
  \lefteqn{\pv(P_{1,3}|known,b) = \alpha \sum_{unknown}\pv(P_{1,3},unknown,known,b)}\\
  &=& \alpha \sum_{unknown}\pv(b|P_{1,3},known,unknown)\pv(P_{1,3},known,unknown)\\
  &=& \alpha \sum_{fringe}\sum_{other} \pv(b|known,P_{1,3},fringe,other)\pv(P_{1,3},known,fringe,other)\\
  &=& \alpha \sum_{fringe}\sum_{other} \pv(b|known,P_{1,3},fringe)\pv(P_{1,3},known,fringe,other)\\
  &=& \alpha \!\sum_{fringe} \!\pv(b|known,P_{1,3},fringe) \sum_{other}\! \pv(P_{1,3},known,fringe,other)\\
  &=&  \alpha \sum_{fringe} \pv(b|known,P_{1,3},fringe)
       \sum_{other} \pv(P_{1,3}) P(known)P(fringe)P(other)\\
  &=&  \alpha\, P(known) \pv(P_{1,3}) \sum_{fringe}
       \pv(b|known,P_{1,3},fringe) P(fringe) \sum_{other} P(other)\\
  &=&  \alpha'\, \pv(P_{1,3}) \sum_{fringe} \pv(b|known,P_{1,3},fringe) P(fringe)
\end{eqnarray*}}

%%%%%%%%%%%% Slide %%%%%%%%%%%%%%%%%%%%%%%%%%%%%%%%%%%%%%%%%%%%%%%%%%%
\heading{Using conditional independence contd.}

\vspace*{0.3in}

\epsfxsize=1.0\textwidth
\fig{\file{figures}{wumpus-fringe-models.ps}}

\mat{\begin{eqnarray*}
\pv(P_{1,3}|known,b) 
   &=&  \alpha'\, \< 0.2(0.04 + 0.16 + 0.16),\ 0.8(0.04 + 0.16) \> \\
   &\approx& \<0.31,0.69\>\\
   &&\\
\pv(P_{2,2}|known,b)  &\approx&  \<0.86,0.14\>
\end{eqnarray*}}


%%%%%%%%%%%% Slide %%%%%%%%%%%%%%%%%%%%%%%%%%%%%%%%%%%%%%%%%%%%%%%%%%%
\heading{Summary}

Probability is a rigorous formalism for uncertain knowledge

\defn{Joint probability distribution} specifies probability of every \defn{atomic event}

Queries can be answered by summing over atomic events

For nontrivial domains, we must find a way to reduce the joint size

\defn{Independence} and \defn{conditional independence} provide the tools










\end{huge} 
\end{document}
