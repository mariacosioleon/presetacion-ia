\documentclass{article}
\usepackage{fleqn}
\usepackage{epsf}
\usepackage[dvips]{color}
\usepackage{aima2e-slides}

\begin{document}

\begin{huge}
\titleslide{Game playing}{Chapter 6}

\sf

%%%%%%%%%%%% Slide %%%%%%%%%%%%%%%%%%%%%%%%%%%%%%%%%%%%%%%%%%%%%%%%%%%
\heading{Outline}

\blob Games

\blob Perfect play\nl
-- minimax decisions\nl
-- $\alpha$--$\beta$ pruning

\blob Resource limits and approximate evaluation

\blob Games of chance

\blob Games of imperfect information


%%%%%%%%%%%% Slide %%%%%%%%%%%%%%%%%%%%%%%%%%%%%%%%%%%%%%%%%%%%%%%%%%%
\heading{Games vs.~search problems}

``Unpredictable'' opponent $\Rightarrow$ solution is a \note{strategy}\\
specifying a move for every possible opponent reply

Time limits $\Rightarrow$ unlikely to find goal, must approximate

Plan of attack:
\begin{itemize}
\item Computer considers possible lines of play (Babbage, 1846)
\item Algorithm for perfect play (Zermelo, 1912; Von Neumann, 1944)
\item Finite horizon, approximate evaluation (Zuse, 1945; Wiener, 1948; \\
      Shannon, 1950)
\item First chess program (Turing, 1951)
\item Machine learning to improve evaluation accuracy (Samuel, 1952--57)
\item Pruning to allow deeper search (McCarthy, 1956)
\end{itemize}


%%%%%%%%%%%% Slide %%%%%%%%%%%%%%%%%%%%%%%%%%%%%%%%%%%%%%%%%%%%%%%%%%%
\heading{Types of games}

\vspace*{0.3in}

\epsfxsize=1.05\textwidth
\fig{\file{figures}{game-types.ps}}




%%%%%%%%%%%% Slide %%%%%%%%%%%%%%%%%%%%%%%%%%%%%%%%%%%%%%%%%%%%%%%%%%%
\heading{Game tree (2-player, deterministic, turns)}


\epsfxsize=0.9\textwidth
\fig{\file{figures}{tictactoe.ps}}

%%%%%%%%%%%% Slide %%%%%%%%%%%%%%%%%%%%%%%%%%%%%%%%%%%%%%%%%%%%%%%%%%%
\heading{Minimax}

Perfect play for deterministic, perfect-information games

Idea: choose move to position with highest \defn{minimax value}\nl
= best achievable payoff against best play

E.g., 2-ply game:

\epsfxsize=0.8\textwidth
\fig{\file{figures}{minimax.ps}}



%%%%%%%%%%%% Slide %%%%%%%%%%%%%%%%%%%%%%%%%%%%%%%%%%%%%%%%%%%%%%%%%%%
\heading{Minimax algorithm}

\input{\file{algorithms}{minimax-algorithm}}


%%%%%%%%%%%% Slide %%%%%%%%%%%%%%%%%%%%%%%%%%%%%%%%%%%%%%%%%%%%%%%%%%%
\heading{Properties of minimax}

\q{Complete} 


%%%%%%%%%%%% Slide %%%%%%%%%%%%%%%%%%%%%%%%%%%%%%%%%%%%%%%%%%%%%%%%%%%
\heading{Properties of minimax}

\q{Complete} Only if tree is finite (chess has specific rules for this).\nl
NB a finite strategy can exist even in an infinite tree!

\q{Optimal} 

%%%%%%%%%%%% Slide %%%%%%%%%%%%%%%%%%%%%%%%%%%%%%%%%%%%%%%%%%%%%%%%%%%
\heading{Properties of minimax}

\q{Complete} Yes, if tree is finite (chess has specific rules for this)

\q{Optimal} Yes, against an optimal opponent. Otherwise??

\q{Time complexity} 


%%%%%%%%%%%% Slide %%%%%%%%%%%%%%%%%%%%%%%%%%%%%%%%%%%%%%%%%%%%%%%%%%%
\heading{Properties of minimax}

\q{Complete} Yes, if tree is finite (chess has specific rules for this)

\q{Optimal} Yes, against an optimal opponent. Otherwise??

\q{Time complexity} \mat{$O(b^m)$}

\q{Space complexity} 

%%%%%%%%%%%% Slide %%%%%%%%%%%%%%%%%%%%%%%%%%%%%%%%%%%%%%%%%%%%%%%%%%%
\heading{Properties of minimax}

\q{Complete} Yes, if tree is finite (chess has specific rules for this)

\q{Optimal} Yes, against an optimal opponent. Otherwise??

\q{Time complexity} \mat{$O(b^m)$}

\q{Space complexity} \mat{$O(bm)$} (depth-first exploration)

For chess, \mat{$b\approx 35$}, \mat{$m \approx 100$} for ``reasonable'' games\nl
$\Rightarrow$ exact solution completely infeasible

But do we need to explore every path?

%%%%%%%%%%%% Slide %%%%%%%%%%%%%%%%%%%%%%%%%%%%%%%%%%%%%%%%%%%%%%%%%%%
\heading{$\alpha$--$\beta$ pruning example}

\vspace*{0.3in}

\epsfxsize=0.9\textwidth
\fig{\sfile{figures}{alpha-beta-progress1.ps}}



%%%%%%%%%%%% Overlay %%%%%%%%%%%%%%%%%%%%%%%%%%%%%%%%%%%%%%%%%%%%%%%%%%%
\pheading{$\alpha$--$\beta$ pruning example}

\vspace*{0.3in}

\epsfxsize=0.9\textwidth
\fig{\sfile{figures}{alpha-beta-progress2.ps}}


%%%%%%%%%%%% Overlay %%%%%%%%%%%%%%%%%%%%%%%%%%%%%%%%%%%%%%%%%%%%%%%%%%%
\pheading{$\alpha$--$\beta$ pruning example}

\vspace*{0.3in}

\epsfxsize=0.9\textwidth
\fig{\sfile{figures}{alpha-beta-progress3.ps}}



%%%%%%%%%%%% Overlay %%%%%%%%%%%%%%%%%%%%%%%%%%%%%%%%%%%%%%%%%%%%%%%%%%%
\pheading{$\alpha$--$\beta$ pruning example}

\vspace*{0.3in}

\epsfxsize=0.9\textwidth
\fig{\sfile{figures}{alpha-beta-progress4.ps}}



%%%%%%%%%%%% Overlay %%%%%%%%%%%%%%%%%%%%%%%%%%%%%%%%%%%%%%%%%%%%%%%%%%%
\pheading{$\alpha$--$\beta$ pruning example}

\vspace*{0.3in}

\epsfxsize=0.9\textwidth
\fig{\sfile{figures}{alpha-beta-progress5.ps}}





%%%%%%%%%%%% Slide %%%%%%%%%%%%%%%%%%%%%%%%%%%%%%%%%%%%%%%%%%%%%%%%%%%
\heading{Why is it called $\alpha$--$\beta$?}

\vspace*{0.1in}

\epsfxsize=0.4\textwidth
\fig{\file{figures}{alpha-beta-general.ps}}

\mat{$\alpha$} is the best value (to {\sc max}) found so far off the current path

If \mat{$V$} is worse than \mat{$\alpha$}, {\sc max} will avoid it
$\Rightarrow$ prune that branch

Define \mat{$\beta$} similarly for {\sc min}







%%%%%%%%%%%% Slide %%%%%%%%%%%%%%%%%%%%%%%%%%%%%%%%%%%%%%%%%%%%%%%%%%%
\heading{The $\alpha$--$\beta$ algorithm}

\vspace*{-0.2in}

\input{\file{algorithms}{alpha-beta-algorithm}}



%%%%%%%%%%%% Slide %%%%%%%%%%%%%%%%%%%%%%%%%%%%%%%%%%%%%%%%%%%%%%%%%%%
\heading{Properties of $\alpha$--$\beta$}

Pruning \emph{does not} affect final result

Good move ordering improves effectiveness of pruning

With ``perfect ordering,'' time complexity = \mat{$O(b^{m/2})$}\nl
$\Rightarrow$ \emph{doubles} solvable depth

A simple example of the value of reasoning about which 
computations are relevant (a form of \note{metareasoning})

Unfortunately, \mat{$35^{50}$} is still impossible!



%%%%%%%%%%%% Slide %%%%%%%%%%%%%%%%%%%%%%%%%%%%%%%%%%%%%%%%%%%%%%%%%%%
\heading{Resource limits}

Standard approach:
\begin{itemize}
\item Use {\sc Cutoff-Test} instead of {\sc Terminal-Test}\nl
e.g., depth limit (perhaps add \defn{quiescence search})
\item Use {\sc Eval} instead of {\sc Utility}\nl
i.e., \defn{evaluation function} that estimates desirability of position
\end{itemize}

Suppose we have \mat{$100$} seconds, explore \mat{$10^4$} nodes/second\nl
$\Rightarrow$ \mat{$10^6$} nodes per move $\approx$ \mat{$35^{8/2}$}\nl
$\Rightarrow$ $\alpha$--$\beta$ reaches depth 8 $\Rightarrow$ pretty good chess program


%%%%%%%%%%%% Slide %%%%%%%%%%%%%%%%%%%%%%%%%%%%%%%%%%%%%%%%%%%%%%%%%%%
\heading{Evaluation functions}

\vspace*{0.2in}

\epsfxsize=0.95\textwidth
\fig{\file{figures}{chess-evaluation-bc.ps}}

For chess, typically \note{linear} weighted sum of \defn{features}
\mat{\[
{\sc Eval}(s) = w_1 f_1(s) + w_2 f_2(s) + \ldots + w_n f_n(s)
\]}
e.g., \mat{$w_1 = 9$} with \\
\mat{$f_1(s)$} = (number of white queens) -- (number of black queens),\ \ etc.




%%%%%%%%%%%% Slide %%%%%%%%%%%%%%%%%%%%%%%%%%%%%%%%%%%%%%%%%%%%%%%%%%%
\heading{Digression: Exact values don't matter}

\vspace*{0.3in}

\epsfxsize=1.05\textwidth
\fig{\file{figures}{ordinal-utility.ps}}

Behaviour is preserved under any \emph{monotonic} transformation of
{\sc Eval}

Only the order matters:\nl
payoff in deterministic games acts as an \defn{ordinal utility} function



%%%%%%%%%%%% Slide %%%%%%%%%%%%%%%%%%%%%%%%%%%%%%%%%%%%%%%%%%%%%%%%%%%
\heading{Deterministic games in practice}

Checkers: Chinook ended 40-year-reign of human world champion Marion
Tinsley in 1994. Used an endgame database defining perfect play for
all positions involving 8 or fewer pieces on the board, a total of
443,748,401,247 positions.

Chess: Deep Blue defeated human world champion Gary Kasparov
in a six-game match in 1997. Deep Blue searches 200 million positions
per second, uses very sophisticated evaluation, and undisclosed methods for
extending some lines of search up to 40 ply.

Othello: human champions refuse to compete against computers, who are
too good.

Go: human champions refuse to compete against computers, who are too
bad. In go, $b > 300$, so most programs use pattern knowledge bases to
suggest plausible moves.




%%%%%%%%%%%% Slide %%%%%%%%%%%%%%%%%%%%%%%%%%%%%%%%%%%%%%%%%%%%%%%%%%%
\heading{Nondeterministic games: backgammon}

\vspace*{0.3in}

\epsfxsize=0.65\maxfigwidth
\fig{\file{figures}{backgammon-position.ps}}




%%%%%%%%%%%% Slide %%%%%%%%%%%%%%%%%%%%%%%%%%%%%%%%%%%%%%%%%%%%%%%%%%%
\heading{Nondeterministic games in general}

In nondeterministic games, chance introduced by dice, card-shuffling

Simplified example with coin-flipping:

\vspace*{0.3in}

\epsfxsize=0.65\textwidth
\fig{\file{figures}{expectiminimax-simple.ps}}


%%%%%%%%%%%% Slide %%%%%%%%%%%%%%%%%%%%%%%%%%%%%%%%%%%%%%%%%%%%%%%%%%%
\heading{Algorithm for nondeterministic games}

{\sc Expectiminimax} gives perfect play

Just like {\sc Minimax}, except we must also handle chance nodes:

$\ldots$\\
\key{if} \var{state} is a {\sc Max} node \key{then}\nl
   \key{return} the highest {\sc ExpectiMinimax-Value} of {\sc Successors}(\var{state})\\
\key{if} \var{state} is a {\sc Min} node \key{then}\nl
   \key{return} the lowest {\sc ExpectiMinimax-Value} of {\sc Successors}(\var{state})\\
\key{if} \var{state} is a chance node \key{then}\nl
   \key{return} average of {\sc ExpectiMinimax-Value} of {\sc Successors}(\var{state})\\
$\ldots$

%%%%%%%%%%%% Slide %%%%%%%%%%%%%%%%%%%%%%%%%%%%%%%%%%%%%%%%%%%%%%%%%%%
\heading{Nondeterministic games in practice}

Dice rolls increase $b$: 21 possible rolls with 2 dice\\
Backgammon $\approx$ 20 legal moves (can be 6,000 with 1-1 roll)
\[
{\rm depth}\ 4 = 20 \times (21 \times 20)^3 \approx 1.2\times 10^9
\]

As depth increases, probability of reaching a given node shrinks\nl
$\Rightarrow$ value of lookahead is diminished

$\alpha$--$\beta$ pruning is much less effective

{\sc TDGammon} uses depth-2 search + very good {\sc Eval}\nl
$\approx$ world-champion level


%%%%%%%%%%%% Slide %%%%%%%%%%%%%%%%%%%%%%%%%%%%%%%%%%%%%%%%%%%%%%%%%%%
\heading{Digression: Exact values DO matter}

\vspace*{0.3in}

\epsfxsize=1.05\textwidth
\fig{\file{figures}{chance-evaluation.ps}}


Behaviour is preserved only by \note{positive linear} transformation of
{\sc Eval}

Hence {\sc Eval} should be proportional to the expected payoff


%%%%%%%%%%%% Slide %%%%%%%%%%%%%%%%%%%%%%%%%%%%%%%%%%%%%%%%%%%%%%%%%%%
\heading{Games of imperfect information}

E.g., card games, where opponent's initial cards are unknown

Typically we can calculate a probability for each possible deal

Seems just like having one big dice roll at the beginning of the game\mat{$^*$}

\note{Idea}: compute the minimax value of each action in each deal,\nl
   then choose the action with highest expected value over all deals\mat{$^*$}

Special case: if an action is optimal for all deals, it's optimal.\mat{$^*$}

GIB, current best bridge program, approximates this idea by\al
1) generating 100 deals consistent with bidding information\al
2) picking the action that wins most tricks on average 


%%%%%%%%%%%% Slide %%%%%%%%%%%%%%%%%%%%%%%%%%%%%%%%%%%%%%%%%%%%%%%%%%%
\heading{Example}

Four-card bridge/whist/hearts hand, {\sc Max} to play first

\vspace*{0.1in}

\epsfxsize=1.08\textwidth
\fig{\sfile{figures}{card-tree1.ps}}

%%%%%%%%%%%% Slide %%%%%%%%%%%%%%%%%%%%%%%%%%%%%%%%%%%%%%%%%%%%%%%%%%%
\heading{Example}

Four-card bridge/whist/hearts hand, {\sc Max} to play first

\vspace*{0.1in}

\epsfxsize=1.08\textwidth
\fig{\sfile{figures}{card-tree2.ps}}

%%%%%%%%%%%% Slide %%%%%%%%%%%%%%%%%%%%%%%%%%%%%%%%%%%%%%%%%%%%%%%%%%%
\heading{Example}

Four-card bridge/whist/hearts hand, {\sc Max} to play first

\vspace*{0.1in}

\epsfxsize=1.08\textwidth
\fig{\sfile{figures}{card-tree3.ps}}

%%%%%%%%%%%% Slide %%%%%%%%%%%%%%%%%%%%%%%%%%%%%%%%%%%%%%%%%%%%%%%%%%%
\heading{Commonsense example}

Road A leads to a small heap of gold pieces\\
Road B leads to a fork:\nl
   take the left fork and you'll find a mound of jewels;\nl
   take the right fork and you'll be run over by a bus.


%%%%%%%%%%%% Slide %%%%%%%%%%%%%%%%%%%%%%%%%%%%%%%%%%%%%%%%%%%%%%%%%%%
\heading{Commonsense example}

Road A leads to a small heap of gold pieces\\
Road B leads to a fork:\nl
   take the left fork and you'll find a mound of jewels;\nl
   take the right fork and you'll be run over by a bus.

Road A leads to a small heap of gold pieces\\
Road B leads to a fork:\nl
   take the left fork and you'll  be run over by a bus;\nl
   take the right fork and you'll find a mound of jewels.

%%%%%%%%%%%% Slide %%%%%%%%%%%%%%%%%%%%%%%%%%%%%%%%%%%%%%%%%%%%%%%%%%%
\heading{Commonsense example}

Road A leads to a small heap of gold pieces\\
Road B leads to a fork:\nl
   take the left fork and you'll find a mound of jewels;\nl
   take the right fork and you'll be run over by a bus.

Road A leads to a small heap of gold pieces\\
Road B leads to a fork:\nl
   take the left fork and you'll  be run over by a bus;\nl
   take the right fork and you'll find a mound of jewels.

Road A leads to a small heap of gold pieces\\
Road B leads to a fork:\nl
   guess correctly and you'll find a mound of jewels;\nl
   guess incorrectly and you'll be run over by a bus.


%%%%%%%%%%%% Slide %%%%%%%%%%%%%%%%%%%%%%%%%%%%%%%%%%%%%%%%%%%%%%%%%%%
\heading{Proper analysis}

\mat{*} Intuition that the value of an action is the average of its values\\
in all actual states is \emph{WRONG}

With partial observability, value of an action depends on the\\
\defn{information state} or \defn{belief state} the agent is in

Can generate and search a tree of information states

Leads to rational behaviors such as\al
\blob Acting to obtain information\al
\blob Signalling to one's partner\al
\blob Acting randomly to minimize information disclosure



%%%%%%%%%%%% Slide %%%%%%%%%%%%%%%%%%%%%%%%%%%%%%%%%%%%%%%%%%%%%%%%%%%
\heading{Summary}

Games are fun to work on! (and dangerous)

They illustrate several important points about AI

\blob perfection is unattainable $\Rightarrow$ must approximate

\blob good idea to think about what to think about

\blob uncertainty constrains the assignment of values to states

\blob optimal decisions depend on information state, not real state

Games are to AI as grand prix racing is to automobile design



\end{huge}
\end{document}







%%%%%%%%%%%% Slide %%%%%%%%%%%%%%%%%%%%%%%%%%%%%%%%%%%%%%%%%%%%%%%%%%%
\heading{Pruning in nondeterministic game trees}

A version of $\alpha$-$\beta$ pruning is possible:

\vspace*{0.3in}

\epsfxsize=0.9\textwidth
\fig{\sfile{figures}{expectiminimax-pruning1.ps}}

%%%%%%%%%%%% Slide %%%%%%%%%%%%%%%%%%%%%%%%%%%%%%%%%%%%%%%%%%%%%%%%%%%
\heading{Pruning in nondeterministic game trees}

A version of $\alpha$-$\beta$ pruning is possible:

\vspace*{0.3in}

\epsfxsize=0.9\textwidth
\fig{\sfile{figures}{expectiminimax-pruning2.ps}}

%%%%%%%%%%%% Slide %%%%%%%%%%%%%%%%%%%%%%%%%%%%%%%%%%%%%%%%%%%%%%%%%%%
\heading{Pruning in nondeterministic game trees}

A version of $\alpha$-$\beta$ pruning is possible:

\vspace*{0.3in}

\epsfxsize=0.9\textwidth
\fig{\sfile{figures}{expectiminimax-pruning3.ps}}

%%%%%%%%%%%% Slide %%%%%%%%%%%%%%%%%%%%%%%%%%%%%%%%%%%%%%%%%%%%%%%%%%%
\heading{Pruning in nondeterministic game trees}

A version of $\alpha$-$\beta$ pruning is possible:

\vspace*{0.3in}

\epsfxsize=0.9\textwidth
\fig{\sfile{figures}{expectiminimax-pruning4.ps}}

%%%%%%%%%%%% Slide %%%%%%%%%%%%%%%%%%%%%%%%%%%%%%%%%%%%%%%%%%%%%%%%%%%
\heading{Pruning in nondeterministic game trees}

A version of $\alpha$-$\beta$ pruning is possible:

\vspace*{0.3in}

\epsfxsize=0.9\textwidth
\fig{\sfile{figures}{expectiminimax-pruning5.ps}}

%%%%%%%%%%%% Slide %%%%%%%%%%%%%%%%%%%%%%%%%%%%%%%%%%%%%%%%%%%%%%%%%%%
\heading{Pruning in nondeterministic game trees}

A version of $\alpha$-$\beta$ pruning is possible:

\vspace*{0.3in}

\epsfxsize=0.9\textwidth
\fig{\sfile{figures}{expectiminimax-pruning6.ps}}

%%%%%%%%%%%% Slide %%%%%%%%%%%%%%%%%%%%%%%%%%%%%%%%%%%%%%%%%%%%%%%%%%%
\heading{Pruning in nondeterministic game trees}

A version of $\alpha$-$\beta$ pruning is possible:

\vspace*{0.3in}

\epsfxsize=0.9\textwidth
\fig{\sfile{figures}{expectiminimax-pruning7.ps}}

%%%%%%%%%%%% Slide %%%%%%%%%%%%%%%%%%%%%%%%%%%%%%%%%%%%%%%%%%%%%%%%%%%
\heading{Pruning in nondeterministic game trees}

A version of $\alpha$-$\beta$ pruning is possible:

\vspace*{0.3in}

\epsfxsize=0.9\textwidth
\fig{\sfile{figures}{expectiminimax-pruning8.ps}}

%%%%%%%%%%%% Slide %%%%%%%%%%%%%%%%%%%%%%%%%%%%%%%%%%%%%%%%%%%%%%%%%%%
\heading{Pruning contd.}

More pruning occurs if we can bound the leaf values

\vspace*{0.3in}

\epsfxsize=0.9\textwidth
\fig{\sfile{figures}{expectiminimax-bounded1.ps}}

%%%%%%%%%%%% Slide %%%%%%%%%%%%%%%%%%%%%%%%%%%%%%%%%%%%%%%%%%%%%%%%%%%
\heading{Pruning contd.}

More pruning occurs if we can bound the leaf values

\vspace*{0.3in}

\epsfxsize=0.9\textwidth
\fig{\sfile{figures}{expectiminimax-bounded2.ps}}

%%%%%%%%%%%% Slide %%%%%%%%%%%%%%%%%%%%%%%%%%%%%%%%%%%%%%%%%%%%%%%%%%%
\heading{Pruning contd.}

More pruning occurs if we can bound the leaf values

\vspace*{0.3in}

\epsfxsize=0.9\textwidth
\fig{\sfile{figures}{expectiminimax-bounded3.ps}}

%%%%%%%%%%%% Slide %%%%%%%%%%%%%%%%%%%%%%%%%%%%%%%%%%%%%%%%%%%%%%%%%%%
\heading{Pruning contd.}

More pruning occurs if we can bound the leaf values

\vspace*{0.3in}

\epsfxsize=0.9\textwidth
\fig{\sfile{figures}{expectiminimax-bounded4.ps}}

%%%%%%%%%%%% Slide %%%%%%%%%%%%%%%%%%%%%%%%%%%%%%%%%%%%%%%%%%%%%%%%%%%
\heading{Pruning contd.}

More pruning occurs if we can bound the leaf values

\vspace*{0.3in}

\epsfxsize=0.9\textwidth
\fig{\sfile{figures}{expectiminimax-bounded5.ps}}

%%%%%%%%%%%% Slide %%%%%%%%%%%%%%%%%%%%%%%%%%%%%%%%%%%%%%%%%%%%%%%%%%%
\heading{Pruning contd.}

More pruning occurs if we can bound the leaf values

\vspace*{0.3in}

\epsfxsize=0.9\textwidth
\fig{\sfile{figures}{expectiminimax-bounded6.ps}}




