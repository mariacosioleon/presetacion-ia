\documentclass{article}
\usepackage{fleqn}
\usepackage{epsf}
\usepackage[dvips]{color}
\usepackage{aima2e-slides}

\begin{document}

\begin{huge}
\titleslide{Intelligent Agents}{Chapter 2}

\sf

%%%%%%%%%%%% Slide %%%%%%%%%%%%%%%%%%%%%%%%%%%%%%%%%%%%%%%%%%%%%%%%%%%
\heading{Reminders}


\emph{Assignment 0 (lisp refresher) due 1/28}

\emph{Lisp/emacs/AIMA tutorial}: 11-1 today and Monday, 271 Soda




%%%%%%%%%%%% Slide %%%%%%%%%%%%%%%%%%%%%%%%%%%%%%%%%%%%%%%%%%%%%%%%%%%
\heading{Outline}

\blob Agents and environments

\blob Rationality

\blob PEAS (Performance measure, Environment, Actuators, Sensors)

\blob Environment types

\blob Agent types

%%%%%%%%%%%% Slide %%%%%%%%%%%%%%%%%%%%%%%%%%%%%%%%%%%%%%%%%%%%%%%%%%%
\heading{Agents and environments}

\epsfxsize=0.65\textwidth
\fig{\file{figures}{agent-environment.ps}}

\defn{Agents} include humans, robots, softbots, thermostats, etc.

The \defn{agent function} maps from percept histories to actions:
\[f: {\cal P}^* \rightarrow {\cal A}\]
The \defn{agent program} runs on the physical \defn{architecture} to produce $f$



%%%%%%%%%%%% Slide %%%%%%%%%%%%%%%%%%%%%%%%%%%%%%%%%%%%%%%%%%%%%%%%%%%
\heading{Vacuum-cleaner world}

\vspace*{0.3in}

\epsfxsize=0.65\textwidth
\fig{\file{figures}{vacuum2-environment.ps}}

Percepts: location and contents, e.g., $[A,Dirty]$

Actions: $Left$, $Right$, $Suck$, $NoOp$

%%%%%%%%%%%% Slide %%%%%%%%%%%%%%%%%%%%%%%%%%%%%%%%%%%%%%%%%%%%%%%%%%%
\heading{A vacuum-cleaner agent}

\begin{mytabular}{@{\extracolsep\fill}|@{\qquad}@{\qquad}l@{\qquad}@{\qquad}@{\qquad}@{\qquad}|@{\qquad}l@{\qquad}|}
\hline
\tabhead Percept sequence & Action \\
\hline
\tabtop 
$[A,Clean]$	& $Right$ \\
$[A,Dirty]$	& $Suck$ \\
$[B,Clean]$	& $Left$ \\
$[B,Dirty]$	& $Suck$ \\
$[A,Clean]$, $[A,Clean]$	& $Right$ \\
$[A,Clean]$, $[A,Dirty]$	& $Suck$ \\
%$[A,Clean]$, $[B,Clean]$	& $Left$ \\
%$[A,Clean]$, $[B,Dirty]$	& $Suck$ \\
%$[A,Dirty]$, $[A,Clean]$	& $Right$ \\
%$[A,Dirty]$, $[A,Dirty]$	& $Suck$ \\
%$\vdots$			& $\vdots$ \\
%$[A,Clean]$, $[A,Clean]$, $[A,Clean]$	& $Right$ \\
%$[A,Clean]$, $[A,Clean]$, $[A,Dirty]$	& $Suck$ \\
$\vdots$				& $\vdots$
\tabbot \\
\hline
\end{mytabular}
%

\medskip

\code{
\func{Reflex-Vacuum-Agent}{{\ts}[\var{location},\var{status}]}{an action}
\bodysep
    \key{if} \var{status} = \var{Dirty} \key{then return} \var{Suck}
    \key{else if} \var{location} = \var{A} \key{then return} \var{Right}
    \key{else if} \var{location} = \var{B} \key{then return} \var{Left}
}


What is the \emph{right} function? \\
Can it be implemented in a small agent program?

%%%%%%%%%%%% Slide %%%%%%%%%%%%%%%%%%%%%%%%%%%%%%%%%%%%%%%%%%%%%%%%%%%
\heading{Rationality}

Fixed \defn{performance measure} evaluates the \note{environment sequence}\al
-- one point per square cleaned up in time $T$?\al
-- one point per clean square per time step, minus one per move?\al
-- penalize for ${}> k$ dirty squares?

A \txr{rational agent} chooses whichever action maximizes the \txr{expected} value of
the performance measure \txr{given the percept sequence to date}

Rational $\neq$ omniscient\nl
   -- percepts may not supply all relevant information\\
Rational $\neq$ clairvoyant\nl
   -- action outcomes may not be as expected\\
Hence, rational $\neq$ successful

Rational $\implies$ exploration, learning, autonomy




%%%%%%%%%%%% Slide %%%%%%%%%%%%%%%%%%%%%%%%%%%%%%%%%%%%%%%%%%%%%%%%%%%
\heading{PEAS}

To design a rational agent, we must specify the \txr{task environment}

Consider, e.g., the task of designing an automated taxi:

\q{Performance measure}

\q{Environment}

\q{Actuators}

\q{Sensors}


%%%%%%%%%%%% Slide %%%%%%%%%%%%%%%%%%%%%%%%%%%%%%%%%%%%%%%%%%%%%%%%%%%
\heading{PEAS}

To design a rational agent, we must specify the \txr{task environment}

Consider, e.g., the task of designing an automated taxi:

\q{Performance measure} safety, destination, profits, legality, comfort, $\ldots$

\q{Environment} US streets/freeways, traffic, pedestrians, weather, $\ldots$

\q{Actuators} steering, accelerator, brake, horn, speaker/display, $\ldots$

\q{Sensors} video, accelerometers, gauges, engine sensors, keyboard, GPS, $\ldots$

%%%%%%%%%%%% Slide %%%%%%%%%%%%%%%%%%%%%%%%%%%%%%%%%%%%%%%%%%%%%%%%%%%
\heading{Internet shopping agent}

\q{Performance measure}

\q{Environment}

\q{Actuators}

\q{Sensors}



%%%%%%%%%%%% Slide %%%%%%%%%%%%%%%%%%%%%%%%%%%%%%%%%%%%%%%%%%%%%%%%%%%
\heading{Internet shopping agent}

\q{Performance measure} price, quality, appropriateness, efficiency

\q{Environment} current and future WWW sites, vendors, shippers

\q{Actuators} display to user, follow URL, fill in form

\q{Sensors} HTML pages (text, graphics, scripts)



%%%%%%%%%%%% Slide %%%%%%%%%%%%%%%%%%%%%%%%%%%%%%%%%%%%%%%%%%%%%%%%%%%
\heading{Environment types}

\begin{mytabular}{@{\extracolsep\fill}|@{\squad}l@{\quad}|cccc@{\squad}|}
\hline
\tabhead & {Solitaire} & {Backgammon} & {Internet shopping} & {Taxi} \\
\hline
\tabtop 
\q{Observable}   &   &   &   &  \\
\q{Deterministic} &   &   &   &  \\
\q{Episodic}      &   &   &   &  \\
\q{Static}       &   &   &   &  \\
\q{Discrete}       &   &   &   &  \\
\tabbot 
\q{Single-agent}     &   &   &   &  \\
\hline
\end{mytabular}


%%%%%%%%%%%% Slide %%%%%%%%%%%%%%%%%%%%%%%%%%%%%%%%%%%%%%%%%%%%%%%%%%%
\heading{Environment types}

\begin{mytabular}{@{\extracolsep\fill}|@{\squad}l@{\quad}|cccc@{\squad}|}
\hline
\tabhead & {Solitaire} & {Backgammon} & {Internet shopping} & {Taxi} \\
\hline
\tabtop 
\q{Observable}   & Yes  & Yes  & No  & No \\
\q{Deterministic}  &   &   &   &  \\
\q{Episodic}     &   &   &   &  \\
\q{Static}           &   &   &   &  \\
\q{Discrete}     &   &   &   &  \\
\tabbot 
\q{Single-agent} &   &   &    &  \\
\hline
\end{mytabular}

%%%%%%%%%%%% Slide %%%%%%%%%%%%%%%%%%%%%%%%%%%%%%%%%%%%%%%%%%%%%%%%%%%
\heading{Environment types}

\begin{mytabular}{@{\extracolsep\fill}|@{\squad}l@{\quad}|cccc@{\squad}|}
\hline
\tabhead & {Solitaire} & {Backgammon} & {Internet shopping} & {Taxi} \\
\hline
\tabtop 
\q{Observable}   & Yes  & Yes  & No  & No \\
\q{Deterministic}  & Yes  & No  & Partly  & No \\
\q{Episodic}     &   &   &   &  \\
\q{Static}           &   &   &   &  \\
\q{Discrete}     &   &   &   &  \\
\tabbot 
\q{Single-agent} &   &   &   &  \\
\hline
\end{mytabular}

%%%%%%%%%%%% Slide %%%%%%%%%%%%%%%%%%%%%%%%%%%%%%%%%%%%%%%%%%%%%%%%%%%
\heading{Environment types}

\begin{mytabular}{@{\extracolsep\fill}|@{\squad}l@{\quad}|cccc@{\squad}|}
\hline
\tabhead & {Solitaire} & {Backgammon} & {Internet shopping} & {Taxi} \\
\hline
\tabtop 
\q{Observable}   & Yes  & Yes  & No  & No \\
\q{Deterministic}  & Yes  & No  & Partly  & No \\
\q{Episodic}     & No  & No  & No  & No \\
\q{Static}           &   &   &   &  \\
\q{Discrete}     &   &   &   &  \\
\tabbot 
\q{Single-agent} &   &   &   &  \\
\hline
\end{mytabular}

%%%%%%%%%%%% Slide %%%%%%%%%%%%%%%%%%%%%%%%%%%%%%%%%%%%%%%%%%%%%%%%%%%
\heading{Environment types}

\begin{mytabular}{@{\extracolsep\fill}|@{\squad}l@{\quad}|cccc@{\squad}|}
\hline
\tabhead & {Solitaire} & {Backgammon} & {Internet shopping} & {Taxi} \\
\hline
\tabtop 
\q{Observable}   & Yes  & Yes  & No  & No \\
\q{Deterministic}  & Yes  & No  & Partly  & No \\
\q{Episodic}     & No  & No  & No  & No \\
\q{Static}           & Yes  & Semi  & Semi  & No \\
\q{Discrete}     &   &   &   &  \\
\tabbot 
\q{Single-agent} &   &   &   &  \\
\hline
\end{mytabular}

%%%%%%%%%%%% Slide %%%%%%%%%%%%%%%%%%%%%%%%%%%%%%%%%%%%%%%%%%%%%%%%%%%
\heading{Environment types}

\begin{mytabular}{@{\extracolsep\fill}|@{\squad}l@{\quad}|cccc@{\squad}|}
\hline
\tabhead & {Solitaire} & {Backgammon} & {Internet shopping} & {Taxi} \\
\hline
\tabtop 
\q{Observable}   & Yes  & Yes  & No  & No \\
\q{Deterministic}  & Yes  & No  & Partly  & No \\
\q{Episodic}     & No  & No  & No  & No \\
\q{Static}           & Yes  & Semi  & Semi  & No \\
\q{Discrete}     & Yes  & Yes  & Yes  & No \\
\tabbot 
\q{Single-agent} &   &   &   &  \\
\hline
\end{mytabular}

%%%%%%%%%%%% Slide %%%%%%%%%%%%%%%%%%%%%%%%%%%%%%%%%%%%%%%%%%%%%%%%%%%
\heading{Environment types}

\begin{mytabular}{@{\extracolsep\fill}|@{\squad}l@{\quad}|cccc@{\squad}|}
\hline
\tabhead & {Solitaire} & {Backgammon} & {Internet shopping} & {Taxi} \\
\hline
\tabtop 
\q{Observable}   & Yes  & Yes  & No  & No \\
\q{Deterministic}  & Yes  & No  & Partly  & No \\
\q{Episodic}     & No  & No  & No  & No \\
\q{Static}           & Yes  & Semi  & Semi  & No \\
\q{Discrete}     & Yes  & Yes  & Yes  & No \\
\tabbot 
\q{Single-agent} & Yes  & No  & Yes (except auctions)  & No \\
\hline
\end{mytabular}

\bigskip

\emph{The environment type largely determines the agent design}

The real world is (of course) partially observable, stochastic, sequential,
dynamic,  continuous, multi-agent

%%%%%%%%%%%% Slide %%%%%%%%%%%%%%%%%%%%%%%%%%%%%%%%%%%%%%%%%%%%%%%%%%%
\heading{Agent types}

Four basic types in order of increasing generality:\al
-- simple reflex agents\al
-- reflex agents with state\al
-- goal-based agents\al
-- utility-based agents

All these can be turned into learning agents


%%%%%%%%%%%% Slide %%%%%%%%%%%%%%%%%%%%%%%%%%%%%%%%%%%%%%%%%%%%%%%%%%%
\heading{Simple reflex agents}

\epsfxsize=0.95\textwidth
\fig{\file{figures}{simple-reflex-agent.ps}}


%%%%%%%%%%%% Slide %%%%%%%%%%%%%%%%%%%%%%%%%%%%%%%%%%%%%%%%%%%%%%%%%%%
\heading{Example}

\medskip

\code{
\func{Reflex-Vacuum-Agent}{{\ts}[\var{location},\var{status}]}{an action}
\bodysep
    \key{if} \var{status} = \var{Dirty} \key{then return} \var{Suck}
    \key{else if} \var{location} = \var{A} \key{then return} \var{Right}
    \key{else if} \var{location} = \var{B} \key{then return} \var{Left}
}


\bigskip

\begin{verbatim}
(setq joe (make-agent :name 'joe :body (make-agent-body)
                      :program (make-reflex-vacuum-agent-program)))

(defun make-reflex-vacuum-agent-program ()
  #'(lambda (percept)
      (let ((location (first percept)) (status (second percept)))
        (cond ((eq status 'dirty) 'Suck)
              ((eq location 'A) 'Right)
              ((eq location 'B) 'Left)))))
\end{verbatim}



%%%%%%%%%%%% Slide %%%%%%%%%%%%%%%%%%%%%%%%%%%%%%%%%%%%%%%%%%%%%%%%%%%
\heading{Reflex agents with state}

\epsfxsize=0.95\textwidth
\fig{\file{figures}{reflex+state-agent.ps}}


%%%%%%%%%%%% Slide %%%%%%%%%%%%%%%%%%%%%%%%%%%%%%%%%%%%%%%%%%%%%%%%%%%
\heading{Example}

\medskip

\code{
\func{Reflex-Vacuum-Agent}{{\ts}[\var{location},\var{status}]}{an action}
\firststatic{last\_A,\ last\_B}{numbers, initially $\infty$}
\bodysep
    \key{if} \var{status} = \var{Dirty} \key{then} $\ldots$
}


\bigskip

\begin{verbatim}
(defun make-reflex-vacuum-agent-with-state-program ()
  (let ((last-A infinity) (last-B infinity))
  #'(lambda (percept)
      (let ((location (first percept)) (status (second percept)))
        (incf last-A) (incf last-B)
        (cond 
         ((eq status 'dirty) 
          (if (eq location 'A) (setq last-A 0) (setq last-B 0))
          'Suck)
         ((eq location 'A) (if (> last-B 3) 'Right 'NoOp))
         ((eq location 'B) (if (> last-A 3) 'Left 'NoOp)))))))
\end{verbatim}



%%%%%%%%%%%% Slide %%%%%%%%%%%%%%%%%%%%%%%%%%%%%%%%%%%%%%%%%%%%%%%%%%%
\heading{Goal-based agents}

\epsfxsize=0.95\textwidth
\fig{\file{figures}{goal-based-agent.ps}}


%%%%%%%%%%%% Slide %%%%%%%%%%%%%%%%%%%%%%%%%%%%%%%%%%%%%%%%%%%%%%%%%%%
\heading{Utility-based agents}

\epsfxsize=0.95\textwidth
\fig{\file{figures}{utility-based-agent.ps}}

%%%%%%%%%%%% Slide %%%%%%%%%%%%%%%%%%%%%%%%%%%%%%%%%%%%%%%%%%%%%%%%%%%
\heading{Learning agents}

\epsfxsize=0.95\textwidth
\fig{\file{figures}{learning-agent.ps}}


%%%%%%%%%%%%%% Slide %%%%%%%%%%%%%%%%%%%%%%%%%%%%%%%%%%%%%%%%%%%%%%%%%%%
%%\heading{AIMA code}
%%
%%The code for each topic is divided into four directories:\al
%%-- {\tt agents}: code defining agent types and programs\al
%%-- {\tt algorithms}: code for the methods used by the agent programs\al
%%-- {\tt environments}: code defining environment types, simulations\al
%%-- {\tt domains}: problem types and instances for input to algorithms\\
%%(Often run algorithms on domains rather than agents in environments.)
%%
%%\begin{verbatim}
%%(setq joe (make-agent :name 'joe :body (make-agent-body)
%%                      :program (make-dumb-agent-program)))
%%
%%(defun make-dumb-agent-program ()
%%  (let ((memory nil))
%%    #'(lambda (percept)
%%        (push percept memory)
%%        'no-op)))
%%\end{verbatim}
%%
%%%%%%%%%%%% Slide %%%%%%%%%%%%%%%%%%%%%%%%%%%%%%%%%%%%%%%%%%%%%%%%%%%
\heading{Summary}

\defn{Agents} interact with \defn{environments} through
\defn{actuators} and \defn{sensors}

The \defn{agent function} describes what the agent does in all circumstances

The \defn{performance measure} evaluates the environment sequence

A \defn{perfectly rational} agent maximizes expected performance 

\defn{Agent programs} implement (some) agent functions

\defn{PEAS} descriptions define task environments

Environments are categorized along several dimensions:\nl
  \defn{observable}? \defn{deterministic}? \defn{episodic}? \defn{static}? 
  \defn{discrete}? \defn{single-agent}?

Several basic agent architectures exist:\nl
  \defn{reflex}, \defn{reflex with state}, \defn{goal-based}, 
  \defn{utility-based}

\end{huge}
\end{document}







