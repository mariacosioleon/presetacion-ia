\documentclass{article}
\usepackage{fleqn}
\usepackage{epsf}
\usepackage{aima2e-slides}

\begin{document}

\begin{huge}
\titleslide{Learning from Observations}{Chapter 18, Sections 1--3}

\sf

%%%%%%%%%%%% Slide %%%%%%%%%%%%%%%%%%%%%%%%%%%%%%%%%%%%%%%%%%%%%%%%%%%
\heading{Outline}

\blob Learning agents

\blob Inductive learning

\blob Decision tree learning

\blob Measuring learning performance


%%%%%%%%%%%% Slide %%%%%%%%%%%%%%%%%%%%%%%%%%%%%%%%%%%%%%%%%%%%%%%%%%%
\heading{Learning}

Learning is essential for unknown environments,\\
i.e., when designer lacks omniscience

Learning is useful as a system construction method,\\
i.e., expose the agent to reality rather than trying to write it down

Learning modifies the agent's decision mechanisms to improve performance


%%%%%%%%%%%% Slide %%%%%%%%%%%%%%%%%%%%%%%%%%%%%%%%%%%%%%%%%%%%%%%%%%%
\heading{Learning agents}

\epsfxsize=0.95\textwidth
\fig{\file{figures}{learning-model.ps}}

% Points to make:
% 1) Performance element is what we have called ``agent'' up to now
% 2) Critic/LearningElement/ProblemGenerator do the ``improving''
% 3) Performance standard is fixed; 
%    a) can't adjust performance standard to flatter own behaviour
%    b) no standard *in the environment*: ordinary chess and
%       suicide chess LOOK identical. Essentially, certain kinds of percepts
%       are ``hardwired'' as good/bad (e.g., pain, hunger)
% 4) Learning element may use knowledge already acquired in the perf. element
% 5) Learning may require experimentation - actions an agent might not
%    normally consider such as dropping rocks for the Tower of Pisa



%%%%%%%%%%%% Slide %%%%%%%%%%%%%%%%%%%%%%%%%%%%%%%%%%%%%%%%%%%%%%%%%%%
\heading{Learning element}

Design of learning element is dictated by\al
\blob what type of performance element is used\al
\blob which functional component is to be learned\al
\blob how that functional compoent is represented\al
\blob what kind of feedback is available\\
Example scenarios:

\vspace*{0.1in}

\epsfxsize=1.07\textwidth
\fig{\file{figures}{learning-elements.ps}}

\defn{Supervised learning}: correct answers for each instance\\
\defn{Reinforcement learning}: occasional rewards

% Points to make:
% 1) Learning transition model is ``supervised'' if observable
% 2) Supervised learning of correct actions requires ``teacher''
% 3) Reinforcement learning is harder, but requires no teacher


%%%%%%%%%%%% Slide %%%%%%%%%%%%%%%%%%%%%%%%%%%%%%%%%%%%%%%%%%%%%%%%%%%
\heading{Inductive learning (a.k.a.~Science)}

Simplest form: learn a function from examples (\emph{tabula rasa})

\mat{$f$} is the \defn{target function}

An \defn{example} is a pair \mat{$x$}, \mat{$f(x)$}, e.g., 
\mat{$\begin{array}{c|c|c}O&O&X \\
                     \hline
                      &X&  \\
                     \hline
                     X& &  \end{array}\ \ ,\ \ +1$}

Problem: find a(n) \defn{hypothesis} \mat{$h$}\nl
   such that \mat{$h \approx f$}\nl
   given a \defn{training set} of examples

(\emph{This is a highly simplified model of real learning:\al
  -- Ignores prior knowledge\al
  -- Assumes a deterministic, observable ``environment''\al
  -- Assumes examples are \emph{given}\al
  -- Assumes that the agent \emph{wants} to learn \mat{$f$}---why?})

% Points to make:
% 1) For Science, the ``agent'' is the whole human race over all time
% 2) tabula rasa = ``blank slate'' (lit. erased writing tablet)



%%%%%%%%%%%% Slide %%%%%%%%%%%%%%%%%%%%%%%%%%%%%%%%%%%%%%%%%%%%%%%%%%%
\heading{Inductive learning method}

Construct/adjust \mat{$h$} to agree with \mat{$f$} on training set\\
(\mat{$h$} is \defn{consistent} if it agrees with \mat{$f$} on all examples)

E.g., curve fitting:

\vspace*{0.2in}

\epsfxsize=0.5\textwidth
\fig{\sfile{figures}{curve-fitting1.ps}}

%%%%%%%%%%%% Slide %%%%%%%%%%%%%%%%%%%%%%%%%%%%%%%%%%%%%%%%%%%%%%%%%%%
\heading{Inductive learning method}

Construct/adjust \mat{$h$} to agree with \mat{$f$} on training set\\
(\mat{$h$} is \defn{consistent} if it agrees with \mat{$f$} on all examples)

E.g., curve fitting:

\vspace*{0.2in}

\epsfxsize=0.5\textwidth
\fig{\sfile{figures}{curve-fitting2.ps}}

%%%%%%%%%%%% Slide %%%%%%%%%%%%%%%%%%%%%%%%%%%%%%%%%%%%%%%%%%%%%%%%%%%
\heading{Inductive learning method}

Construct/adjust \mat{$h$} to agree with \mat{$f$} on training set\\
(\mat{$h$} is \defn{consistent} if it agrees with \mat{$f$} on all examples)

E.g., curve fitting:

\vspace*{0.2in}

\epsfxsize=0.5\textwidth
\fig{\sfile{figures}{curve-fitting3.ps}}

%%%%%%%%%%%% Slide %%%%%%%%%%%%%%%%%%%%%%%%%%%%%%%%%%%%%%%%%%%%%%%%%%%
\heading{Inductive learning method}

Construct/adjust \mat{$h$} to agree with \mat{$f$} on training set\\
(\mat{$h$} is \defn{consistent} if it agrees with \mat{$f$} on all examples)

E.g., curve fitting:

\vspace*{0.2in}

\epsfxsize=0.5\textwidth
\fig{\sfile{figures}{curve-fitting4.ps}}

%%%%%%%%%%%% Slide %%%%%%%%%%%%%%%%%%%%%%%%%%%%%%%%%%%%%%%%%%%%%%%%%%%
\heading{Inductive learning method}

Construct/adjust \mat{$h$} to agree with \mat{$f$} on training set\\
(\mat{$h$} is \defn{consistent} if it agrees with \mat{$f$} on all examples)

E.g., curve fitting:

\vspace*{0.2in}

\epsfxsize=0.5\textwidth
\fig{\sfile{figures}{curve-fitting5.ps}}

%%%%%%%%%%%% Slide %%%%%%%%%%%%%%%%%%%%%%%%%%%%%%%%%%%%%%%%%%%%%%%%%%%
\heading{Inductive learning method}

Construct/adjust \mat{$h$} to agree with \mat{$f$} on training set\\
(\mat{$h$} is \defn{consistent} if it agrees with \mat{$f$} on all examples)

E.g., curve fitting:

\vspace*{0.2in}

\epsfxsize=0.5\textwidth
\fig{\sfile{figures}{curve-fitting5.ps}}

\defn{Ockham's razor}: maximize a combination of consistency and simplicity




%%%%%%%%%%%% Slide %%%%%%%%%%%%%%%%%%%%%%%%%%%%%%%%%%%%%%%%%%%%%%%%%%%
\heading{Attribute-based representations}

Examples described by \defn{attribute values} (Boolean, discrete, continuous, etc.)\\
E.g., situations where I will/won't wait for a table:

\input{\file{tables}{restaurant-example-table}}%

\defn{Classification} of examples is \defn{positive} (T) or \defn{negative} (F)

% Points to make:
% 1) Give detailed attribute definitions (on p.532)


%%%%%%%%%%%% Slide %%%%%%%%%%%%%%%%%%%%%%%%%%%%%%%%%%%%%%%%%%%%%%%%%%%
\heading{Decision trees}

One possible representation for hypotheses\\
E.g., here is the ``true'' tree for deciding whether to wait:

\vspace*{0.2in}

\epsfxsize=0.8\textwidth
\fig{\file{figures}{restaurant-tree.ps}}

% Points to make:
% 1) Explain how the decision tree evaluates an example, e.g., \mat{$X_3$}
% 2) Some of the original set of attributes are irrelevant
% 3) This case is REALIZABLE - i.e., target definable in hypothesis class

%%%%%%%%%%%% Slide %%%%%%%%%%%%%%%%%%%%%%%%%%%%%%%%%%%%%%%%%%%%%%%%%%%
\heading{Expressiveness}

Decision trees can express any function of the input attributes.\\
E.g., for Boolean functions, truth table row \mat{$\rightarrow$} path to leaf:

\vspace*{0.2in}

\epsfxsize=0.65\textwidth
\fig{\file{figures}{xor-decision-tree.ps}}

Trivially, there is a consistent decision tree for any training set\\
w/ one path to leaf for each example (unless \mat{$f$} nondeterministic in \mat{$x$})\\
but it probably won't generalize to new examples

Prefer to find more \emph{compact} decision trees


%%%%%%%%%%%% Slide %%%%%%%%%%%%%%%%%%%%%%%%%%%%%%%%%%%%%%%%%%%%%%%%%%%
\heading{Hypothesis spaces}

\q{How many distinct decision trees with \mat{$n$} Boolean attributes}

% Points to make: get students to think about question before
% going on to next slides which have the answer

%%%%%%%%%%%% Slide %%%%%%%%%%%%%%%%%%%%%%%%%%%%%%%%%%%%%%%%%%%%%%%%%%%
\heading{Hypothesis spaces}

\q{How many distinct decision trees with \mat{$n$} Boolean attributes}

= number of Boolean functions

%%%%%%%%%%%% Slide %%%%%%%%%%%%%%%%%%%%%%%%%%%%%%%%%%%%%%%%%%%%%%%%%%%
\heading{Hypothesis spaces}

\q{How many distinct decision trees with \mat{$n$} Boolean attributes}

= number of Boolean functions\\
= number of distinct truth tables with \mat{$2^n$} rows

%%%%%%%%%%%% Slide %%%%%%%%%%%%%%%%%%%%%%%%%%%%%%%%%%%%%%%%%%%%%%%%%%%
\heading{Hypothesis spaces}

\q{How many distinct decision trees with \mat{$n$} Boolean attributes}

= number of Boolean functions\\
= number of distinct truth tables with \mat{$2^n$} rows = \mat{$2^{2^n}$}

%%%%%%%%%%%% Slide %%%%%%%%%%%%%%%%%%%%%%%%%%%%%%%%%%%%%%%%%%%%%%%%%%%
\heading{Hypothesis spaces}

\q{How many distinct decision trees with \mat{$n$} Boolean attributes}

= number of Boolean functions\\
= number of distinct truth tables with \mat{$2^n$} rows = \mat{$2^{2^n}$}

E.g., with 6 Boolean attributes, there are 18,446,744,073,709,551,616 trees

%%%%%%%%%%%% Slide %%%%%%%%%%%%%%%%%%%%%%%%%%%%%%%%%%%%%%%%%%%%%%%%%%%
\heading{Hypothesis spaces}

\q{How many distinct decision trees with \mat{$n$} Boolean attributes}

= number of Boolean functions\\
= number of distinct truth tables with \mat{$2^n$} rows = \mat{$2^{2^n}$}

E.g., with 6 Boolean attributes, there are 18,446,744,073,709,551,616 trees

\q{How many purely conjunctive hypotheses (e.g., \mat{$Hungry\land\lnot Rain$})}

% Points to make: get students to think about question before
% going on to next slides which have the answer

%%%%%%%%%%%% Slide %%%%%%%%%%%%%%%%%%%%%%%%%%%%%%%%%%%%%%%%%%%%%%%%%%%
\heading{Hypothesis spaces}

\q{How many distinct decision trees with \mat{$n$} Boolean attributes}

= number of Boolean functions\\
= number of distinct truth tables with \mat{$2^n$} rows = \mat{$2^{2^n}$}

E.g., with 6 Boolean attributes, there are 18,446,744,073,709,551,616 trees

\q{How many purely conjunctive hypotheses (e.g., \mat{$Hungry\land\lnot Rain$})}

Each attribute can be in (positive), in (negative), or out\nl
\mat{$\implies$} \mat{$3^n$} distinct conjunctive hypotheses

More expressive hypothesis space\al
 -- increases chance that target function can be expressed \quad\smiley\al
 -- increases number of hypotheses consistent w/ training set\nl
    \mat{$\implies$} may get worse predictions\quad\frowny


%%%%%%%%%%%% Slide %%%%%%%%%%%%%%%%%%%%%%%%%%%%%%%%%%%%%%%%%%%%%%%%%%%
\heading{Decision tree learning}

Aim: find a small tree consistent with the training examples

Idea: (recursively) choose ``most significant'' attribute as root of (sub)tree

\code{
\func{DTL}{\var{examples{\ac}attributes{\ac}default}}{a decision tree}
\bodysep
    \key{if} \var{examples} is empty \key{then} \key{return} \var{default}
    \key{else} \key{if} all \var{examples} have the same classification \key{then} \key{return} the classification
    \key{else} \key{if} \var{attributes} is empty \key{then} \key{return} \prog{Mode}(\var{examples})
    \key{else} 
          \setq{\var{best}}{\prog{Choose-Attribute}(\var{attributes}{\ac}\var{examples})}
          \setq{\var{tree}}{a new decision tree with root test \var{best}}
          \key{for} \key{each} value \var{v}$_i$ of \var{best} \key{do}
                \setq{\var{examples}$_i$}{$\{$elements of \var{examples} with $\var{best} = v_i\}$}
                \setq{\v{subtree}}{}\prog{DTL}(\var{examples}$_i{\ac}$\var{attributes}$\,-\,$\var{best}{\ac}\prog{Mode}(\var{examples}))
                add a branch to \var{tree} with label \var{v}$_i$ and subtree \var{subtree}
          \key{return} \var{tree}
}


% Points to make:
% 1) Explain inputs
% 2) Explain recursive step -- pick attribute, subdivide examples
% 3) Explain base cases: 
%    a) empty examples - arises for empty branches of non-Boolean parent attribute
%    b) uniform example classification - this is ``normal'' leaf
%    c) empty attributes - target is not deterministic in input attributes

%%%%%%%%%%%% Slide %%%%%%%%%%%%%%%%%%%%%%%%%%%%%%%%%%%%%%%%%%%%%%%%%%%
\heading{Choosing an attribute}

Idea: a good attribute splits the examples into subsets that are
(ideally) ``all positive'' or ``all negative''

\vspace*{0.5in}

\epsfxsize=1.0\textwidth
\fig{\file{figures}{restaurant-roots.ps}}

\mat{$Patrons?$} is a better choice---gives \emph{information} about the classification



%%%%%%%%%%%% Slide %%%%%%%%%%%%%%%%%%%%%%%%%%%%%%%%%%%%%%%%%%%%%%%%%%%
\heading{Information}

Information answers questions

The more clueless I am about the answer initially, the more
information is contained in the answer

Scale: 1 bit = answer to Boolean question with prior \mat{$\<0.5,0.5\>$}

Information in an answer when prior is \mat{$\<P_1,\ldots,P_n\>$} is
\mat{\[
  H(\<P_1,\ldots,P_n\>) = \mysum_{i\eq 1}^n - P_i \log_2 P_i
\]}
(also called \defn{entropy} of the prior)



%%%%%%%%%%%% Slide %%%%%%%%%%%%%%%%%%%%%%%%%%%%%%%%%%%%%%%%%%%%%%%%%%%
\heading{Information contd.}

Suppose we have \mat{$p$} positive and \mat{$n$} negative examples at the root\nl
\mat{$\implies$} \mat{$H(\<p/(p+n),n/(p+n)\>)$} bits needed to classify a new example\\
E.g., for 12 restaurant examples, \mat{$p\eq n\eq 6$} so we need 1 bit

An attribute splits the examples \mat{$E$} into subsets \mat{$E_i$}, each of which
(we hope) needs less information to complete the classification

Let \mat{$E_i$} have \mat{$p_i$} positive and \mat{$n_i$} negative examples\al
\mat{$\implies$} \mat{$H(\<p_i/(p_i+n_i),n_i/(p_i+n_i)\>)$} bits needed to classify a new example\al
\mat{$\implies$} \emph{expected} number of bits per example over all branches is
\mat{\[
  \mysum_i \ \ \frac{p_i+n_i}{p+n} \ H(\<p_i/(p_i+n_i),n_i/(p_i+n_i)\>)
\]}
For \mat{$Patrons?$}, this is 0.459 bits, for \mat{$Type$} this is (still) 1 bit

\mat{$\implies$} choose the attribute that minimizes the remaining information needed


%%%%%%%%%%%% Slide %%%%%%%%%%%%%%%%%%%%%%%%%%%%%%%%%%%%%%%%%%%%%%%%%%%
\heading{Example contd.}

Decision tree learned from the 12 examples:

\vspace*{0.2in}

\epsfxsize=0.6\textwidth
\fig{\file{figures}{induced-restaurant-tree.ps}}

Substantially simpler than ``true'' tree---a more complex hypothesis
isn't justified by small amount of data



%%%%%%%%%%%% Slide %%%%%%%%%%%%%%%%%%%%%%%%%%%%%%%%%%%%%%%%%%%%%%%%%%%
\heading{Performance measurement}

How do we know that \mat{$h\approx f$}? (Hume's \emph{Problem of Induction})

1) Use theorems of computational/statistical learning theory

2) Try \mat{$h$} on a new \defn{test set} of examples \al
  (use \emph{same distribution over example space} as training set)

\defn{Learning curve} = \% correct on test set as a function of training set size

\vspace*{-0.2in}

\epsfxsize=0.6\textwidth
\fig{\file{graphs}{restaurant-dtl-curve.ps}}




% Points to make:
% 1) Restaurant data; graph averaged over 20 trials



%%%%%%%%%%%% Slide %%%%%%%%%%%%%%%%%%%%%%%%%%%%%%%%%%%%%%%%%%%%%%%%%%%
\heading{Performance measurement contd.}

Learning curve depends on\al
 -- \defn{realizable} (can express target function) vs. \defn{non-realizable}\nl
     non-realizability can be due to missing attributes \nl
     or restricted hypothesis class (e.g., thresholded linear function)\al
 -- redundant expressiveness (e.g., loads of irrelevant attributes)

\vspace*{0.2in}

\epsfxsize=0.95\textwidth
\fig{\file{figures}{learning-curves.ps}}



%%%%%%%%%%%% Slide %%%%%%%%%%%%%%%%%%%%%%%%%%%%%%%%%%%%%%%%%%%%%%%%%%%
\heading{Summary}

Learning needed for unknown environments, lazy designers

Learning agent = performance element + learning element

Learning method depends on type of performance element, available\\
feedback, type of component to be improved, and its representation

For supervised learning, the aim is to find a simple hypothesis\\
that is approximately consistent with training examples

Decision tree learning using information gain 

Learning performance = prediction accuracy measured on test set

\end{huge}
\end{document}



