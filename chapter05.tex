\documentclass{article}
\usepackage{fleqn}
\usepackage{epsf}
\usepackage[dvips]{color}
\usepackage{aima2e-slides}

\begin{document}

\begin{huge}
\titleslide{Constraint Satisfaction Problems}{Chapter 5}

\sf

%%%%%%%%%%%% Slide %%%%%%%%%%%%%%%%%%%%%%%%%%%%%%%%%%%%%%%%%%%%%%%%%%%
\heading{Outline}

\blob CSP examples

\blob Backtracking search for CSPs

\blob Problem structure and problem decomposition

\blob Local search for CSPs




%%%%%%%%%%%% Slide %%%%%%%%%%%%%%%%%%%%%%%%%%%%%%%%%%%%%%%%%%%%%%%%%%%
\heading{Constraint satisfaction problems (CSPs)}

Standard search problem:\al
\note{state} is a ``black box''---any old data structure\nl
that supports goal test, eval, successor

CSP:\al
\note{state} is defined by \defn{variables} \mat{$X_i$}
   with \defn{values} from \defn{domain} \mat{$D_i$}\al
\ \al
\note{goal test} is a set of \defn{constraints} specifying\nl
   allowable combinations of values for subsets of variables

Simple example of a \emph{formal representation language}

Allows useful \emph{general-purpose} algorithms with more power\\
than standard search algorithms


%%%%%%%%%%%% Slide %%%%%%%%%%%%%%%%%%%%%%%%%%%%%%%%%%%%%%%%%%%%%%%%%%%
\heading{Example: Map-Coloring}

\vspace*{-0.0in}

\epsfxsize=0.65\textwidth
\fig{\file{figures}{australia.ps}}

\vspace*{-0.6in}

\note{Variables} \mat{$WA$}, \mat{$NT$}, \mat{$Q$}, \mat{$NSW$}, \mat{$V$}, \mat{$SA$}, \mat{$T$} \\
\note{Domains} \mat{$D_i = \{red,green,blue\}$}\\
\note{Constraints}: adjacent regions must have different colors\al
e.g., \mat{$WA\neq NT$} (if the language allows this), or\al
\mat{$(WA,NT) \in \{(red,green),(red,blue),(green,red),(green,blue),\ldots\}$}

%%%%%%%%%%%% Slide %%%%%%%%%%%%%%%%%%%%%%%%%%%%%%%%%%%%%%%%%%%%%%%%%%%
\heading{Example: Map-Coloring contd.}

\vspace*{0.1in}

\epsfxsize=0.65\textwidth
\fig{\file{figures}{australia-solution.ps}}

\note{Solutions} are assignments satisfying all constraints, e.g.,\\
\mat{$\{WA\eq red,NT\eq green,Q\eq red,NSW\eq green,V\eq red,SA\eq blue,T\eq green\}$}



%%%%%%%%%%%% Slide %%%%%%%%%%%%%%%%%%%%%%%%%%%%%%%%%%%%%%%%%%%%%%%%%%%
\heading{Constraint graph}

\defn{Binary CSP}: each constraint relates at most two variables

\defn{Constraint graph}: nodes are variables, arcs show constraints

\vspace*{-0.5in}

\epsfxsize=0.6\textwidth
\fig{\file{figures}{australia-csp.ps}}

General-purpose CSP algorithms use the graph structure\\
to speed up search. E.g., Tasmania is an independent subproblem!

%%%%%%%%%%%% Slide %%%%%%%%%%%%%%%%%%%%%%%%%%%%%%%%%%%%%%%%%%%%%%%%%%%
\heading{Varieties of CSPs}

Discrete variables\al
  finite domains; size \mat{$d$} $\implies$ \mat{$O(d^n)$} complete assignments\nl
    \blob e.g., Boolean CSPs, incl.~Boolean satisfiability (NP-complete)\al
  infinite domains (integers, strings, etc.)\nl
    \blob e.g., job scheduling, variables are start/end days for each job\nl
    \blob need a \defn{constraint language}, e.g., \mat{$StartJob_1 + 5 \leq StartJob_3$}\nl
    \blob \note{linear} constraints solvable, \note{nonlinear} undecidable

Continuous variables\al
    \blob e.g., start/end times for Hubble Telescope observations\al
    \blob linear constraints solvable in poly time by LP methods

%%%%%%%%%%%% Slide %%%%%%%%%%%%%%%%%%%%%%%%%%%%%%%%%%%%%%%%%%%%%%%%%%%
\heading{Varieties of constraints}

\defn{Unary} constraints involve a single variable, \al
   e.g., \mat{$SA\neq green$}

\defn{Binary} constraints involve pairs of variables, \al
   e.g., \mat{$SA\neq WA$}

\defn{Higher-order} constraints involve 3 or more variables,\al
   e.g., cryptarithmetic column constraints

\defn{Preferences} (soft constraints), e.g., \mat{$red$} is better than \mat{$green$}\\
often representable by a cost for each variable assignment\al
$\rightarrow$ constrained optimization problems


%%%%%%%%%%%% Slide %%%%%%%%%%%%%%%%%%%%%%%%%%%%%%%%%%%%%%%%%%%%%%%%%%%
\heading{Example: Cryptarithmetic}

\vspace*{0.3in}

\epsfxsize=0.85\textwidth
\fig{\file{figures}{cryptarithmetic.ps}}

\note{Variables}: \mat{$F\ T\ U\ W\ R\ O\ X_1\ X_2\ X_3$}\\
\note{Domains}: \mat{$\{0,1,2,3,4,5,6,7,8,9\}$}\\
\note{Constraints}\al
\mat{$\alldiff(F,T,U,W,R,O)$}\al
\mat{$O + O = R + 10\cdot X_1$}, etc.



%%%%%%%%%%%% Slide %%%%%%%%%%%%%%%%%%%%%%%%%%%%%%%%%%%%%%%%%%%%%%%%%%%
\heading{Real-world CSPs}

Assignment problems\al
    e.g., who teaches what class

Timetabling problems\al
    e.g., which class is offered when and where?

Hardware configuration

Spreadsheets

Transportation scheduling

Factory scheduling

Floorplanning

\vspace*{0.3in}

Notice that many real-world problems involve real-valued variables


%%%%%%%%%%%% Slide %%%%%%%%%%%%%%%%%%%%%%%%%%%%%%%%%%%%%%%%%%%%%%%%%%%
\heading{Standard search formulation (incremental)}

Let's start with the straightforward, dumb approach, then fix it

States are defined by the values assigned so far

\blob \note{Initial state}: the empty assignment, \mat{$\emptyset$}

\blob \note{Successor function}: assign a value to an unassigned variable\nl
that does not conflict with current assignment.\nl
$\implies$ fail if no legal assignments (not fixable!)

\blob \note{Goal test}: the current assignment is complete

\vspace*{0.3in}

1) This is the same for all CSPs! \smiley\\
2) Every solution appears at depth \mat{$n$} with \mat{$n$} variables\nl
   $\implies$ use depth-first search\\
3) Path is irrelevant, so can also use complete-state formulation\\
4) \mat{$b\eq (n-\ell)d$} at depth \mat{$\ell$}, hence \mat{$n!d^n$} leaves!!!! \frowny



%%%%%%%%%%%% Slide %%%%%%%%%%%%%%%%%%%%%%%%%%%%%%%%%%%%%%%%%%%%%%%%%%%
\heading{Backtracking search}

Variable assignments are \defn{commutative}, i.e.,\nl
 [\mat{$WA\eq red$} then \mat{$NT\eq green$}]\ \ same as \ \ [\mat{$NT\eq green$} then \mat{$WA\eq red$}]

Only need to consider assignments to a single variable at each node\nl
   $\implies$ \mat{$b \eq d$} and there are \mat{$d^n$} leaves\nl

Depth-first search for CSPs with single-variable assignments\\
is called \defn{backtracking} search

Backtracking search is the basic uninformed algorithm for CSPs

Can solve \mat{$n$}-queens for \mat{$n \approx 25$}


%%%%%%%%%%%% Slide %%%%%%%%%%%%%%%%%%%%%%%%%%%%%%%%%%%%%%%%%%%%%%%%%%%
\heading{Backtracking search}

\code{
\func{Backtracking-Search}{\var{csp}}{solution/failure}
    \key{return} \prog{Recursive-Backtracking}(\(\{\,\}\){\ac}\var{csp})
\bodysep
\func{Recursive-Backtracking}{\var{assignment}{\ac}\var{csp}}{soln/failure}
    \key{if} \var{assignment} is complete \key{then return} \var{assignment}
    \setq{\var{var}}{\noprog{Select-Unassigned-Variable}(\noprog{Variables}[\var{csp}]{\ac}\var{assignment}{\ac}\var{csp})}
    \key{for} \key{each} \var{value} \key{in} \noprog{Order-Domain-Values}(\var{var}{\ac}\var{assignment}{\ac}\var{csp}) \key{do}
          \key{if} \var{value} is consistent with \var{assignment} given \noprog{Constraints}[\var{csp}] \key{then}
                add \{\var{var} = \var{value}\} to \var{assignment}
                \setq{\var{result}}{\prog{Recursive-Backtracking}(\var{assignment}{\ac}\var{csp})}
                \key{if} \var{result} \(\neq\) \var{failure} \key{then} \key{return} \var{result}
                remove \{\var{var} = \var{value}\} from \var{assignment}
    \key{return} \var{failure}
}


%%%%%%%%%%%% Slide %%%%%%%%%%%%%%%%%%%%%%%%%%%%%%%%%%%%%%%%%%%%%%%%%%%
\heading{Backtracking example}

\epsfxsize=0.95\maxfigwidth
\fig{\sfile{figures}{backtrack-progress1.ps}}

%%%%%%%%%%%% Slide %%%%%%%%%%%%%%%%%%%%%%%%%%%%%%%%%%%%%%%%%%%%%%%%%%%
\heading{Backtracking example}

\epsfxsize=0.95\maxfigwidth
\fig{\sfile{figures}{backtrack-progress2.ps}}

%%%%%%%%%%%% Slide %%%%%%%%%%%%%%%%%%%%%%%%%%%%%%%%%%%%%%%%%%%%%%%%%%%
\heading{Backtracking example}

\epsfxsize=0.95\maxfigwidth
\fig{\sfile{figures}{backtrack-progress3.ps}}

%%%%%%%%%%%% Slide %%%%%%%%%%%%%%%%%%%%%%%%%%%%%%%%%%%%%%%%%%%%%%%%%%%
\heading{Backtracking example}

\epsfxsize=0.95\maxfigwidth
\fig{\sfile{figures}{backtrack-progress4.ps}}

%%%%%%%%%%%% Slide %%%%%%%%%%%%%%%%%%%%%%%%%%%%%%%%%%%%%%%%%%%%%%%%%%%
\heading{Improving backtracking efficiency}

\emph{General-purpose} methods can give huge gains in speed:
\begin{enumerate}
\item Which variable should be assigned next?
\item In what order should its values be tried?
\item Can we detect inevitable failure early?
\item Can we take advantage of problem structure?
\end{enumerate}

%%%%%%%%%%%% Slide %%%%%%%%%%%%%%%%%%%%%%%%%%%%%%%%%%%%%%%%%%%%%%%%%%%
\heading{Minimum remaining values}

Minimum remaining values (MRV):\al
   choose the variable with the fewest legal values

\vspace*{0.3in}

\epsfxsize=1.0\maxfigwidth
\fig{\file{figures}{australia-most-constrained-variable.ps}}

%%%%%%%%%%%% Slide %%%%%%%%%%%%%%%%%%%%%%%%%%%%%%%%%%%%%%%%%%%%%%%%%%%
\heading{Degree heuristic}

Tie-breaker among MRV variables

Degree heuristic:\al
   choose the variable with the most constraints on remaining variables

\vspace*{0.3in}

\epsfxsize=1.0\maxfigwidth
\fig{\file{figures}{australia-most-constraining-variable.ps}}

%%%%%%%%%%%% Slide %%%%%%%%%%%%%%%%%%%%%%%%%%%%%%%%%%%%%%%%%%%%%%%%%%%
\heading{Least constraining value}

Given a variable, choose the least constraining value:\al
   the one that rules out the fewest values in the remaining variables

\vspace*{0.3in}

\epsfxsize=1.05\maxfigwidth
\fig{\file{figures}{australia-least-constraining-value.ps}}

Combining these heuristics makes 1000 queens feasible




%%%%%%%%%%%% Slide %%%%%%%%%%%%%%%%%%%%%%%%%%%%%%%%%%%%%%%%%%%%%%%%%%%
\heading{Forward checking}

\note{Idea}: Keep track of remaining legal values for unassigned variables\\
\phantom{\note{Idea}: }Terminate search when any variable has no legal values

\vspace*{0.1in}

\epsfxsize=1.0\maxfigwidth
\fig{\sfile{figures}{forward-checking-progress1.ps}}


%%%%%%%%%%%% Slide %%%%%%%%%%%%%%%%%%%%%%%%%%%%%%%%%%%%%%%%%%%%%%%%%%%
\heading{Forward checking}

\note{Idea}: Keep track of remaining legal values for unassigned variables\\
\phantom{\note{Idea}: }Terminate search when any variable has no legal values

\vspace*{0.1in}

\epsfxsize=1.0\maxfigwidth
\fig{\sfile{figures}{forward-checking-progress2.ps}}


%%%%%%%%%%%% Slide %%%%%%%%%%%%%%%%%%%%%%%%%%%%%%%%%%%%%%%%%%%%%%%%%%%
\heading{Forward checking}

\note{Idea}: Keep track of remaining legal values for unassigned variables\\
\phantom{\note{Idea}: }Terminate search when any variable has no legal values

\vspace*{0.1in}

\epsfxsize=1.0\maxfigwidth
\fig{\sfile{figures}{forward-checking-progress3.ps}}


%%%%%%%%%%%% Slide %%%%%%%%%%%%%%%%%%%%%%%%%%%%%%%%%%%%%%%%%%%%%%%%%%%
\heading{Forward checking}

\note{Idea}: Keep track of remaining legal values for unassigned variables\\
\phantom{\note{Idea}: }Terminate search when any variable has no legal values

\vspace*{0.1in}

\epsfxsize=1.0\maxfigwidth
\fig{\sfile{figures}{forward-checking-progress4.ps}}

%%%%%%%%%%%% Slide %%%%%%%%%%%%%%%%%%%%%%%%%%%%%%%%%%%%%%%%%%%%%%%%%%%
\heading{Constraint propagation}

Forward checking propagates information from assigned to unassigned variables,
but doesn't provide early detection for all failures:

\vspace*{0.1in}

\epsfxsize=0.8\maxfigwidth
\fig{\sfile{figures}{forward-checking-progress3.ps}}

\mat{$NT$} and \mat{$SA$} cannot both be blue!

\defn{Constraint propagation} repeatedly enforces constraints locally

%%%%%%%%%%%% Slide %%%%%%%%%%%%%%%%%%%%%%%%%%%%%%%%%%%%%%%%%%%%%%%%%%%
\heading{Arc consistency}

Simplest form of propagation makes each arc \defn{consistent}

\mat{$X\rightarrow Y$} is consistent iff\al
  for \emph{every} value \mat{$x$} of \mat{$X$} there is \emph{some} allowed \mat{$y$}

\vspace*{0.1in}

\epsfxsize=0.8\maxfigwidth
\fig{\sfile{figures}{ac-example1.ps}}

%%%%%%%%%%%% Slide %%%%%%%%%%%%%%%%%%%%%%%%%%%%%%%%%%%%%%%%%%%%%%%%%%%
\heading{Arc consistency}

Simplest form of propagation makes each arc \defn{consistent}

\mat{$X\rightarrow Y$} is consistent iff\al
  for \emph{every} value \mat{$x$} of \mat{$X$} there is \emph{some} allowed \mat{$y$}

\vspace*{0.1in}

\epsfxsize=0.8\maxfigwidth
\fig{\sfile{figures}{ac-example2.ps}}

%%%%%%%%%%%% Slide %%%%%%%%%%%%%%%%%%%%%%%%%%%%%%%%%%%%%%%%%%%%%%%%%%%
\heading{Arc consistency}

Simplest form of propagation makes each arc \defn{consistent}

\mat{$X\rightarrow Y$} is consistent iff\al
  for \emph{every} value \mat{$x$} of \mat{$X$} there is \emph{some} allowed \mat{$y$}

\vspace*{0.1in}

\epsfxsize=0.8\maxfigwidth
\fig{\sfile{figures}{ac-example3.ps}}

If \mat{$X$} loses a value, neighbors of \mat{$X$} need to be rechecked

%%%%%%%%%%%% Slide %%%%%%%%%%%%%%%%%%%%%%%%%%%%%%%%%%%%%%%%%%%%%%%%%%%
\heading{Arc consistency}

Simplest form of propagation makes each arc \defn{consistent}

\mat{$X\rightarrow Y$} is consistent iff\al
  for \emph{every} value \mat{$x$} of \mat{$X$} there is \emph{some} allowed \mat{$y$}

\vspace*{0.1in}

\epsfxsize=0.8\maxfigwidth
\fig{\sfile{figures}{ac-example4.ps}}

If \mat{$X$} loses a value, neighbors of \mat{$X$} need to be rechecked

Arc consistency detects failure earlier than forward checking

Can be run as a preprocessor or after each assignment




%%%%%%%%%%%% Slide %%%%%%%%%%%%%%%%%%%%%%%%%%%%%%%%%%%%%%%%%%%%%%%%%%%
\heading{Arc consistency algorithm}

\code{
\func{AC-3}{{\ts}\var{csp}}{the CSP, possibly with reduced domains}
    \firstinputs{\var{csp}}{a binary CSP with variables \(\{X_{1}, X_{2}, \ldots, X_{n}\}\)}
    \firstlocal{\var{queue}}{a queue of arcs, initially all the arcs in \var{csp}}
\bodysep
    \key{while} \var{queue} is not empty \key{do}
        \setq{\((X_{i}, X_{j})\)}{\prog{Remove-First}(\var{queue})}
        \key{if} \noprog{Remove-Inconsistent-Values}(\(X_{i}, X_{j}\)) \key{then} 
            \key{for each} \(X_{k}\) \key{in} \prog{Neighbors}[\(X_{i}\)] \key{do}
                add (\(X_{k}, X_{i}\)) to \var{queue}
\fnsep
\func{Remove-Inconsistent-Values}{{\ts}\(X_{i}, X_{j}\)}{true iff succeeds}
    \setq{\var{removed}}{\var{false}}
    \key{for each} \var{x} \key{in} \prog{Domain}[\(X_{i}\)] \key{do}
        \key{if} no value \(y\) in \prog{Domain}[\(X_{j}\)] allows (\var{x},\var{y}) to satisfy the constraint \(X_i \leftrightarrow X_j\)
            \key{then} delete \var{x} from \prog{Domain}[\(X_{i}\)];  \setq{\var{removed}}{\var{true}}
    \key{return} \var{removed}
}


\mat{$O(n^2d^3)$}, can be reduced to \mat{$O(n^2d^2)$}
(but detecting \emph{all} is NP-hard)

%%%%%%%%%%%% Slide %%%%%%%%%%%%%%%%%%%%%%%%%%%%%%%%%%%%%%%%%%%%%%%%%%%
\heading{Problem structure}

\vspace*{-0.4in}

\epsfxsize=0.6\textwidth
\fig{\file{figures}{australia-csp.ps}}

Tasmania and mainland are \defn{independent subproblems}

Identifiable as \defn{connected components} of constraint graph

%%%%%%%%%%%% Slide %%%%%%%%%%%%%%%%%%%%%%%%%%%%%%%%%%%%%%%%%%%%%%%%%%%
\heading{Problem structure contd.}

Suppose each subproblem has \mat{$c$} variables out of \mat{$n$} total

Worst-case solution cost is \mat{$n/c \cdot d^c$}, \emph{linear} in \mat{$n$}

E.g., \mat{$n\eq 80$}, \mat{$d\eq 2$}, \mat{$c\eq 20$}\al
  \mat{$2^{80}$} = 4 billion years at 10 million nodes/sec\al
  \mat{$4\cdot 2^{20}$} = 0.4 seconds at 10 million nodes/sec


%%%%%%%%%%%% Slide %%%%%%%%%%%%%%%%%%%%%%%%%%%%%%%%%%%%%%%%%%%%%%%%%%%
\heading{Tree-structured CSPs}

\vspace*{0.3in}

\epsfxsize=0.5\textwidth
\fig{\file{figures}{tree-csp1.ps}}

\note{Theorem}: if the constraint graph has no loops, the CSP can be solved in 
\mat{$O(n\,d^2)$} time

Compare to general CSPs, where worst-case time is \mat{$O(d^n)$}

This property also applies to logical and probabilistic reasoning:\\
an important example of the relation between syntactic restrictions\\
and the complexity of reasoning.



%%%%%%%%%%%% Slide %%%%%%%%%%%%%%%%%%%%%%%%%%%%%%%%%%%%%%%%%%%%%%%%%%%
\heading{Algorithm for tree-structured CSPs}

1. Choose a variable as root, order variables from root to leaves\\
such that every node's parent precedes it in the ordering 

\vspace*{0.2in}

\epsfxsize=0.95\textwidth
\fig{\file{figures}{tree-csp2.ps}}

2. For \mat{$j$} from \mat{$n$} down to \mat{$2$}, apply \prog{RemoveInconsistent}\mat{\mat{$(Parent(X_j),X_j)$}}

3. For \mat{$j$} from \mat{$1$} to \mat{$n$}, assign \mat{$X_j$} consistently with \mat{$Parent(X_j)$}

%%%%%%%%%%%% Slide %%%%%%%%%%%%%%%%%%%%%%%%%%%%%%%%%%%%%%%%%%%%%%%%%%%
\heading{Nearly tree-structured CSPs}

\defn{Conditioning}: instantiate a variable, prune its neighbors' domains

\vspace*{0.2in}

\epsfxsize=0.75\textwidth
\fig{\file{figures}{australia-cutset.ps}}



\defn{Cutset conditioning}: instantiate (in all ways) a set of variables\\ 
such that the remaining constraint graph is a tree

Cutset size \mat{$c$} $\implies$ runtime \mat{$O(d^c\cdot (n-c)d^2)$}, very fast for small \mat{$c$}



%%%%%%%%%%%% Slide %%%%%%%%%%%%%%%%%%%%%%%%%%%%%%%%%%%%%%%%%%%%%%%%%%%
\heading{Iterative algorithms for CSPs}

Hill-climbing, simulated annealing typically work with\\
``complete'' states, i.e., all variables assigned

To apply to CSPs:\al
allow states with unsatisfied constraints\al
operators \emph{reassign} variable values

Variable selection: randomly select any conflicted variable

Value selection by \defn{min-conflicts} heuristic:\al
choose value that violates the fewest constraints\al
i.e., hillclimb with \mat{$h(n)$} = total number of violated constraints


%%%%%%%%%%%% Slide %%%%%%%%%%%%%%%%%%%%%%%%%%%%%%%%%%%%%%%%%%%%%%%%%%%
\heading{Example: 4-Queens}

\note{States}: 4 queens in 4 columns (\mat{$4^4 = 256$} states)

\note{Operators}: move queen in column

\note{Goal test}: no attacks

\note{Evaluation}: \mat{$h(n)$} = number of attacks

\vspace*{0.3in}

\epsfxsize=0.7\textwidth
\epsffile{\file{figures}{4queens-iterative.ps}}

%%%%%%%%%%%% Slide %%%%%%%%%%%%%%%%%%%%%%%%%%%%%%%%%%%%%%%%%%%%%%%%%%%
\heading{Performance of min-conflicts}

Given random initial state, can solve \mat{$n$}-queens in almost constant time for
arbitrary \mat{$n$} with high probability (e.g., \mat{$n$} = 10,000,000)

The same appears to be true for any randomly-generated CSP\\
\emph{except} in a narrow range of the ratio
\[
R = \frac{\mbox{number of constraints}}{\mbox{number of variables}}
\]

\vspace*{0.2in}

\epsfxsize=0.6\textwidth
\fig{\file{figures}{random-csp-runtime.ps}}





%%%%%%%%%%%% Slide %%%%%%%%%%%%%%%%%%%%%%%%%%%%%%%%%%%%%%%%%%%%%%%%%%%
\heading{Summary}

CSPs are a special kind of problem:\al
states defined by values of a fixed set of variables\al
goal test defined by \defn{constraints} on variable values

Backtracking = depth-first search with one variable assigned per node

Variable ordering and value selection heuristics help significantly

Forward checking prevents assignments that guarantee later failure

Constraint propagation (e.g., arc consistency) does additional work\\
to constrain values and detect inconsistencies

The CSP representation allows analysis of problem structure

Tree-structured CSPs can be solved in linear time

Iterative min-conflicts is usually effective in practice







\end{huge}
\end{document}







%%%%%%%%%%%% Slide %%%%%%%%%%%%%%%%%%%%%%%%%%%%%%%%%%%%%%%%%%%%%%%%%%%
\heading{Example: 4-Queens as a CSP}

Assume one queen in each column. Which row does each one go in?

\begin{tabular}{lr}
\hbox{\begin{minipage}[b]{0.6\textwidth}
\note{Variables} \mat{$Q_1$}, \mat{$Q_2$}, \mat{$Q_3$}, \mat{$Q_4$} \\

\note{Domains} \mat{$D_i = \{1,2,3,4\}$}\\


\note{Constraints}\al
\mat{$Q_i \neq Q_j$} (cannot be in same row)\al
\mat{$|Q_i - Q_j| \neq |i-j|$} (or same diagonal)

\end{minipage}}
&
\epsfxsize=2.3in
\epsffile{\file{figures}{4queens.ps}}
\end{tabular}


Translate each constraint into set of allowable values for its variables

E.g., values for \mat{$(Q_1,Q_2)$} are 
\mat{$(1,3)\ (1,4)\ (2,4)\ (3,1)\ (4,1)\ (4,2)$}





