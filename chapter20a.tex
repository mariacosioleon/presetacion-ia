\documentclass{article}
\usepackage{fleqn}
\usepackage{epsf}
\usepackage{aima2e-slides}

\begin{document}

\begin{huge}
\titleslide{Statistical learning}{Chapter 20, Sections 1--3}

\sf

%%%%%%%%%%%% Slide %%%%%%%%%%%%%%%%%%%%%%%%%%%%%%%%%%%%%%%%%%%%%%%%%%%
\heading{Outline}

\blob Bayesian learning

\blob Maximum {\em a posteriori} and maximum likelihood learning

\blob Bayes net learning\nl
-- ML parameter learning with complete data\nl
-- linear regression


%%%%%%%%%%%% Slide %%%%%%%%%%%%%%%%%%%%%%%%%%%%%%%%%%%%%%%%%%%%%%%%%%%
\heading{Full Bayesian learning}

View learning as Bayesian updating of a probability distribution\\
over the \defn{hypothesis space}

\mat{$H$} is the hypothesis variable, values \mat{$h_1,h_2,\ldots$}, prior \mat{$\pv(H)$}

\mat{$j$}th observation \mat{$d_j$} gives the outcome of random variable \mat{$D_j$}\\
training data \mat{$\d \eq d_1,\ldots,d_{\Ncount}$}


Given the data so far, each hypothesis has a posterior probability:
\mat{\[
  P(h_i|\d) = \alpha P(\d|h_i) P(h_i)
\]}%
where \mat{$P(\d|h_i)$} is called the \defn{likelihood}

Predictions use a likelihood-weighted average over the hypotheses:
\mat{\[
  \pv(X|\d) = \mysum_i\ \pv(X|\d,h_i) P(h_i|\d) = \mysum_i\ \pv(X|h_i) P(h_i|\d)
\]}%
No need to pick one best-guess hypothesis!


%%%%%%%%%%%% Slide %%%%%%%%%%%%%%%%%%%%%%%%%%%%%%%%%%%%%%%%%%%%%%%%%%%
\heading{Example}

Suppose there are five kinds of bags of candies:\al
10\% are \mat{$h_1$}: 100\% cherry candies\al
20\% are \mat{$h_2$}: 75\% cherry candies + 25\% lime candies\al
40\% are \mat{$h_3$}: 50\% cherry candies + 50\% lime candies\al
20\% are \mat{$h_4$}: 25\% cherry candies + 75\% lime candies\al
10\% are \mat{$h_5$}: 100\% lime candies

\vspace*{0.2in}

\epsfxsize=0.6\textwidth
\fig{\file{figures}{candy-kinds.ps}}

Then we observe candies drawn from some bag: \epsfxsize=0.3\textwidth \epsffile{\file{figures}{candy-obs.ps}}

What kind of bag is it? What flavour will the next candy be?


%%%%%%%%%%%% Slide %%%%%%%%%%%%%%%%%%%%%%%%%%%%%%%%%%%%%%%%%%%%%%%%%%%
\heading{Posterior probability of hypotheses}

\vspace*{0.2in}

\epsfxsize=0.9\textwidth
\fig{\file{graphs}{limes-bayes-post.eps}}




%%%%%%%%%%%% Slide %%%%%%%%%%%%%%%%%%%%%%%%%%%%%%%%%%%%%%%%%%%%%%%%%%%
\heading{Prediction probability}

\vspace*{0.2in}

\epsfxsize=0.9\textwidth
\fig{\file{graphs}{limes-bayes-pred.eps}}


%%%%%%%%%%%% Slide %%%%%%%%%%%%%%%%%%%%%%%%%%%%%%%%%%%%%%%%%%%%%%%%%%%
\heading{MAP approximation}

Summing over the hypothesis space is often intractable\\
(e.g., 18,446,744,073,709,551,616 Boolean functions of 6 attributes)

\defn{Maximum a posteriori} (MAP) learning: choose \mat{$h_{{\rm MAP}}$} maximizing \mat{$P(h_i|\d)$}

I.e., maximize \mat{$P(\d|h_i)P(h_i)$} or \mat{$\log P(\d|h_i) + \log P(h_i)$}

Log terms can be viewed as (negative of)\al
\note{bits to encode data given hypothesis} + \note{bits to encode hypothesis} \\
This is the basic idea of \defn{minimum description length} (MDL) learning

For deterministic hypotheses, \mat{$P(\d|h_i)$} is 1 if consistent, 0 otherwise\nl
$\implies$ MAP = simplest consistent hypothesis (cf.~science)



%%%%%%%%%%%% Slide %%%%%%%%%%%%%%%%%%%%%%%%%%%%%%%%%%%%%%%%%%%%%%%%%%%
\heading{ML approximation}

For large data sets, prior becomes irrelevant

\defn{Maximum likelihood} (ML) learning: choose \mat{$h_{{\rm ML}}$} maximizing \mat{$P(\d|h_i)$}

I.e., simply get the best fit to the data; identical to MAP for uniform prior\\
(which is reasonable if all hypotheses are of the same complexity)

ML is the ``standard'' (non-Bayesian) statistical learning method


%%%%%%%%%%%% Slide %%%%%%%%%%%%%%%%%%%%%%%%%%%%%%%%%%%%%%%%%%%%%%%%%%%
\heading{ML parameter learning in Bayes nets}

Bag from a new manufacturer; fraction \mat{$\theta$} of cherry candies?\hfill\epsfxsize=1.25in\raisebox{-1.25in}[0pt][0pt]{\epsffile{\file{figures}{ml-network1.ps}}} \\
Any \mat{$\theta$} is possible: continuum of hypotheses \mat{$h_{\theta}$}\\
\mat{$\theta$} is a \defn{parameter} for this simple (\note{binomial}) family of models 

Suppose we unwrap \(\Ncount\) candies, \(c\) cherries and \(\ell\eq \Ncount-c\) limes\\
These are \defn{i.i.d.} (independent, identically distributed) observations, so
\mat{\[
  P(\data |h_{\theta}) = \prod_{j\eq 1}^{\Ncount} P(\datum_j|h_{\theta}) =
  \theta^c\cdot (1-\theta)^{\ell}
\]}%
Maximize this w.r.t.~\mat{$\theta$}---which is easier for the \defn{log-likelihood}:
\mat{\begin{eqnarray*}
  L(\data |h_{\theta}) &=& \log P(\data |h_{\theta}) = \sum_{j\eq 1}^{\Ncount} \log P(\datum_j|h_{\theta}) =
  c\log\theta +  \ell\log(1-\theta) \\
  \frac{dL(\data |h_{\theta})}{d\theta} &=& \frac{c}{\theta} -
  \frac{\ell}{1-\theta} = 0 \qquad \implies \quad \theta =
  \frac{c}{c+\ell} = \frac{c}{\Ncount}
\end{eqnarray*}}%
Seems sensible, but causes problems with 0 counts!








%%%%%%%%%%%% Slide %%%%%%%%%%%%%%%%%%%%%%%%%%%%%%%%%%%%%%%%%%%%%%%%%%%
\heading{Multiple parameters}

Red/green wrapper depends probabilistically on flavor: \hspace*{0.75in}\epsfxsize=2.25in\raisebox{-2.5in}[0pt][0pt]{\epsffile{\file{figures}{ml-network2.ps}}} 

Likelihood  for, e.g., cherry candy in green wrapper:
\mat{\begin{eqnarray*}
\lefteqn{P(\J{F}\eq \J{cherry},\J{W}\eq \J{green}|h_{\theta,\theta_1,\theta_2})}\\
 & = & P(\J{F}\eq \J{cherry}|h_{\theta,\theta_1,\theta_2})P(\J{W}\eq
 \J{green}|\J{F}\eq \J{cherry},h_{\theta,\theta_1,\theta_2}) \\
 & = & \theta \cdot (1-\theta_1)
\end{eqnarray*}}%
\(\Ncount\) candies, \(r_c\) red-wrapped cherry candies, etc.:
\mat{\begin{eqnarray*}
 P(\data|h_{\theta,\theta_1,\theta_2}) &=&
  \theta^c (1-\theta)^{\ell} \cdot \theta_1^{r_c}(1-\theta_1)^{g_c}
  \cdot \theta_2^{r_{\ell}}(1-\theta_2)^{g_{\ell}}
\end{eqnarray*}}%
\mat{\begin{eqnarray*}
L &=&  [c\log \theta + \ell\log (1-\theta) ]  \\
  &+&   [r_c\log \theta_1 + g_c\log(1-\theta_1) ]  \\
  &+&   [r_{\ell}\log \theta_2 + g_{\ell}\log(1-\theta_2) ]
\end{eqnarray*}}%

%%%%%%%%%%%% Slide %%%%%%%%%%%%%%%%%%%%%%%%%%%%%%%%%%%%%%%%%%%%%%%%%%%
\heading{Multiple parameters contd.}

Derivatives of \mat{$L$} contain only the relevant parameter:
\mat{\[\begin{array}{rclcl}
\displaystyle \frac{\partial L}{\partial\theta} &=& \displaystyle \frac{c}{\theta} - \displaystyle\frac{\ell}{1-\theta} = 0                  & \qquad \implies & \theta = \displaystyle \frac{c}{c+\ell} 
\end{array}\]}%
\mat{\[\begin{array}{rclcl}
\displaystyle \frac{\partial L}{\partial\theta_1} &=& \displaystyle\frac{r_c}{\theta_1} - \displaystyle\frac{g_c}{1-\theta_1} = 0           & \qquad \implies & \theta_1 = \displaystyle\frac{r_c}{r_c+g_c} 
\end{array}\]}%
\mat{\[\begin{array}{rclcl}
\displaystyle \frac{\partial L}{\partial\theta_2} &=& \displaystyle\frac{r_{\ell}}{\theta_2} - \displaystyle\frac{g_{\ell}}{1-\theta_2} = 0 & \qquad \implies & \theta_2 = \displaystyle\frac{r_{\ell}}{r_{\ell}+g_{\ell}}
\end{array}\]}%
With \defn{complete data}, \emph{parameters can be learned separately}




%%%%%%%%%%%% Slide %%%%%%%%%%%%%%%%%%%%%%%%%%%%%%%%%%%%%%%%%%%%%%%%%%%
\heading{Example: linear Gaussian model}

\twofig{\file{graphs}{regression-model.eps}}{\file{graphs}{regression-sample.eps}}

Maximizing \mat{$\displaystyle P(y|x) = \frac{1}{\sqrt{2 \pi} \sigma} e^{-\frac{(y-(\theta_1 x+\theta_2))^2}{2\sigma^2}}$} w.r.t. \mat{$\theta_1$}, \mat{$\theta_2$}

= minimizing \mat{$\displaystyle E = \sum_{j\eq 1}^{\Ncount} (y_j-(\theta_1 x_j+\theta_2))^2$}

That is, minimizing the sum of squared errors gives the ML solution\\
for a linear fit \emph{assuming Gaussian noise of fixed variance}



%%%%%%%%%%%% Slide %%%%%%%%%%%%%%%%%%%%%%%%%%%%%%%%%%%%%%%%%%%%%%%%%%%
\heading{Summary}

Full Bayesian learning gives best possible predictions but is intractable

MAP learning balances complexity with accuracy on training data 

Maximum likelihood assumes uniform prior, OK for large data sets

1. Choose a parameterized family of models to describe the data\\
\note{{\em requires substantial insight and sometimes new models}}

2. Write down the likelihood of the data as a function of the parameters\\
\note{{\em may require summing over hidden variables, i.e., inference}}

3. Write down the derivative of the log likelihood w.r.t. each parameter

4. Find the parameter values such that the derivatives are zero\\
\note{{\em may be hard/impossible; modern optimization techniques help}}



\end{huge}
\end{document}


