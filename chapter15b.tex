\documentclass{article}
\usepackage{fleqn}
\usepackage{epsf}
\usepackage{aima2e-slides}

\begin{document}

\begin{huge}
\titleslide{Speech recognition (briefly)}{Chapter 15, Section 6}

\sf

%%%%%%%%%%%% Slide %%%%%%%%%%%%%%%%%%%%%%%%%%%%%%%%%%%%%%%%%%%%%%%%%%%
\heading{Outline}

\blob Speech as probabilistic inference

\blob Speech sounds

\blob Word pronunciation

\blob Word sequences


%%%%%%%%%%%% Slide %%%%%%%%%%%%%%%%%%%%%%%%%%%%%%%%%%%%%%%%%%%%%%%%%%%
\heading{Speech as probabilistic inference}

\emph{It's not easy to wreck a nice beach}

Speech signals are noisy, variable, ambiguous

What is the \emph{most likely} word sequence, given the speech signal?\al
I.e., choose \mat{$Words$} to maximize \mat{$P(Words|signal)$}

Use Bayes' rule:
\mat{\[
P(Words|signal) = \alpha P(signal|Words) P(Words)
\]}%
I.e., decomposes into \defn{acoustic model} + \defn{language model}

\mat{$Words$} are the hidden state sequence, \mat{$signal$} is the observation sequence

%%%%%%%%%%%% Slide %%%%%%%%%%%%%%%%%%%%%%%%%%%%%%%%%%%%%%%%%%%%%%%%%%%
\heading{Phones}

All human speech is composed from 40-50 \defn{phones}, determined by the\\
configuration of \defn{articulators} (lips, teeth, tongue, vocal cords, air flow)

Form an intermediate level of hidden states between words and signal\nl
  \mat{$\Rightarrow$} acoustic model = pronunciation model + phone model

ARPAbet designed for American English

\newcommand{\bU}[1]{{\bf\underline{#1}}}%bold underline
\begin{mytabular}{@{\extracolsep{\fill}}|@{\sq{0.05}}l@{\sq{0.05}}l@{\sq{0.05}}|@{\sq{0.05}}l@{\sq{0.05}}l@{\sq{0.05}}|@{\sq{0.05}}l@{\sq{0.05}}l@{\sq{0.05}}|}
\hline
\tabtop {[iy]} &  b\bU{ea}t &     [b]   & \bU{b}et &      [p]  &  \bU{p}et \\
{[ih]} &  b\bU{i}t &      [ch]  & \bU{Ch}et &     [r]  &  \bU{r}at \\     
{[ey]} &  b\bU{e}t &      [d]   & \bU{d}ebt &     [s]  &  \bU{s}et \\
{[ao]} &  b\bU{ough}t &   [hh]  & \bU{h}at &      [th] &  \bU{th}ick \\   
{[ow]} &  b\bU{oa}t &     [hv]  & \bU{h}igh&      [dh] &  \bU{th}at \\
{[er]} &  B\bU{er}t &     [l]   & \bU{l}et &      [w]  &  \bU{w}et \\
{[ix]} &  ros\bU{e}s &    [ng]  & si\bU{ng} &  [en]          &butt\bU{on} \\
$\ \ \vdots$ & $\ \ \vdots$ & $\ \ \vdots$ & $\ \ \vdots$ & $\ \ \vdots$ & $\ \ \vdots$ \\
\hline
\end{mytabular}




E.g., ``ceiling'' is [s iy l ih ng] / [s iy l ix ng] / [s iy l en]


%%%%%%%%%%%% Slide %%%%%%%%%%%%%%%%%%%%%%%%%%%%%%%%%%%%%%%%%%%%%%%%%%%
\heading{Speech sounds}

Raw signal is the microphone displacement as a function of time;\\
processed into overlapping 30ms \defn{frames}, each described by \defn{features}

\vspace*{0.2in}

\epsfxsize=0.9\textwidth
\fig{\file{figures}{sr-acoustic-frames.ps}}

Frame features are typically \defn{formants}---peaks in the power spectrum







%%%%%%%%%%%% Slide %%%%%%%%%%%%%%%%%%%%%%%%%%%%%%%%%%%%%%%%%%%%%%%%%%%
\heading{Phone models}

Frame features in \mat{$P(features|phone)$} summarized by\al
  -- an integer in \mat{$[0\ldots 255]$} (using \defn{vector quantization}); or\al
  -- the parameters of a mixture of Gaussians

\defn{Three-state phones}: each phone has three phases (Onset, Mid, End)\al
 E.g., [t] has silent Onset, explosive Mid, hissing End\al
 $\Rightarrow$ \mat{$P(features|phone,phase)$}

\defn{Triphone context}: each phone becomes \mat{$n^2$} distinct phones,
depending on the phones to its left and right\al
  E.g., [t] in ``star'' is written [t(s,aa)] (different from ``tar''!)

Triphones useful for handling \defn{coarticulation} effects: the articulators
have inertia and cannot switch instantaneously between positions\al
E.g., [t] in ``eighth'' has tongue against front teeth

%%%%%%%%%%%% Slide %%%%%%%%%%%%%%%%%%%%%%%%%%%%%%%%%%%%%%%%%%%%%%%%%%%
\heading{Phone model example}

\vspace*{0.2in}

\epsfxsize=0.8\textwidth
\fig{\file{figures}{sr-hmm.ps}}





%%%%%%%%%%%% Slide %%%%%%%%%%%%%%%%%%%%%%%%%%%%%%%%%%%%%%%%%%%%%%%%%%%
\heading{Word pronunciation models}

Each word is described as a distribution over phone sequences

Distribution represented as an HMM transition model

\vspace*{0.2in}

\epsfxsize=0.8\textwidth
\fig{\file{figures}{sr-tomato-b.ps}}

\mat{\begin{formula}
P([towmeytow]|\mbox{``tomato''}) = P([towmaatow]|\mbox{``tomato''}) = 0.1 \\
P([tahmeytow]|\mbox{``tomato''}) = P([tahmaatow]|\mbox{``tomato''}) = 0.4
\end{formula}}

Structure is created manually, transition probabilities learned from data



%%%%%%%%%%%% Slide %%%%%%%%%%%%%%%%%%%%%%%%%%%%%%%%%%%%%%%%%%%%%%%%%%%
\heading{Isolated words}

Phone models + word models fix likelihood \mat{$P(e_{1:t}|word)$} for \defn{isolated word}
\mat{\[
P(word|e_{1:t}) = \alpha P(e_{1:t}|word) P(word)
\]}%
Prior probability \mat{$P(word)$} obtained simply by counting word frequencies

\mat{$P(e_{1:t}|word)$} can be computed recursively: define
\mat{\[
  \bell_{1:t}\eq \pv(\X_t,\e_{1:t})
\]}%
and use the recursive update
\mat{\[
  \bell_{1:t+1} = \noprog{Forward}(\ell_{1:t},\e_{t+1})
\]}%
and then \mat{$P(e_{1:t}|word) = \mysum_{\sx_t} \bell_{1:t}(\x_t)$}

Isolated-word dictation systems with training reach 95--99\% accuracy


%%%%%%%%%%%% Slide %%%%%%%%%%%%%%%%%%%%%%%%%%%%%%%%%%%%%%%%%%%%%%%%%%%
\heading{Continuous speech}

Not just a sequence of isolated-word recognition problems!\al
-- Adjacent words highly correlated\al
-- Sequence of most likely words $\neq$ most likely sequence of words\al
-- Segmentation: there are few gaps in speech\al
-- Cross-word coarticulation---e.g., ``next thing''

Continuous speech systems manage 60--80\% accuracy on a good day


%%%%%%%%%%%% Slide %%%%%%%%%%%%%%%%%%%%%%%%%%%%%%%%%%%%%%%%%%%%%%%%%%%
\heading{Language model}

Prior probability of a word sequence is given by chain rule:
\mat{\[
P(w_{1}\cdots w_{n}) = \prod_{i=1}^{n} P(w_i|w_{1}\cdots w_{i-1})
\]}%
\defn{Bigram model}:
\mat{\[
   P(w_i|w_{1}\cdots w_{i-1}) \approx P(w_i|w_{i-1})
\]}%
Train by counting all word pairs in a large text corpus

More sophisticated models (trigrams, grammars, etc.) help a little bit




%%%%%%%%%%%% Slide %%%%%%%%%%%%%%%%%%%%%%%%%%%%%%%%%%%%%%%%%%%%%%%%%%%
\heading{Combined HMM}

States of the combined language+word+phone model are labelled by\\
the word we're in + the phone in that word + the phone state in that phone

Viterbi algorithm finds the most likely \emph{phone state} sequence

Does segmentation by considering all possible word sequences and boundaries

Doesn't always give the most likely word sequence because\\
each word sequence is the sum over many state sequences

Jelinek invented A$^*$ in 1969 a way to find most likely word sequence\al
   where ``step cost'' is \mat{$-\log P(w_i|w_{i-1})$}





%%%%%%%%%%%% Slide %%%%%%%%%%%%%%%%%%%%%%%%%%%%%%%%%%%%%%%%%%%%%%%%%%%
\heading{DBNs for speech recognition}

\vspace{0.5in}

\epsfxsize=1.1\textwidth
\fig{\file{figures}{speech-dbn.ps}}

Also easy to add variables for, e.g., gender, accent, speed.\\
Zweig and Russell (1998) show up to 40\% error reduction over HMMs

%%%%%%%%%%%% Slide %%%%%%%%%%%%%%%%%%%%%%%%%%%%%%%%%%%%%%%%%%%%%%%%%%%
\heading{Summary}

Since the mid-1970s, speech recognition has been formulated as probabilistic inference

Evidence = speech signal, hidden variables = word and phone sequences

``Context'' effects (coarticulation etc.) are handled by augmenting state

Variability in human speech (speed, timbre, etc., etc.) and background noise\\
make continuous speech recognition in real settings an open problem
  



\end{huge} 
\end{document}

