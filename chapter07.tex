\documentclass{article}
\usepackage{fleqn}
\usepackage{epsf}
\usepackage[dvips]{color}
\usepackage{aima2e-slides}

\begin{document}

\begin{huge}
\titleslide{Logical agents}{Chapter 7}

\sf

%%%%%%%%%%%% Slide %%%%%%%%%%%%%%%%%%%%%%%%%%%%%%%%%%%%%%%%%%%%%%%%%%%
\heading{Outline}

\blob Knowledge-based agents

\blob Wumpus world

\blob Logic in general---models and entailment

\blob Propositional (Boolean) logic

\blob Equivalence, validity, satisfiability

\blob Inference rules and theorem proving\nl
 -- forward chaining\nl
 -- backward chaining\nl
 -- resolution

%\blob Model checking
%
%\blob Propositional knowledge-based agents
%
%\blob Boolean circuit agents


%%%%%%%%%%%% Slide %%%%%%%%%%%%%%%%%%%%%%%%%%%%%%%%%%%%%%%%%%%%%%%%%%%
\heading{Knowledge bases}


\epsfxsize=0.8\maxfigwidth
\fig{\file{figures}{kbs.ps}}

\defn{Knowledge base} = set of \defn{sentences} in a \emph{formal} language

\defn{Declarative} approach to building an agent (or other system):\nl
   \defprog{Tell} it what it needs to know

Then it can \defprog{Ask} itself what to do---answers should follow from the KB

Agents can be viewed at the \defn{knowledge level}\nl
   i.e., \emph{what they know}, regardless of how implemented

Or at the \defn{implementation level}\nl
   i.e., data structures in KB and algorithms that manipulate them


%%%%%%%%%%%% Slide %%%%%%%%%%%%%%%%%%%%%%%%%%%%%%%%%%%%%%%%%%%%%%%%%%%
\heading{A simple knowledge-based agent}

\input{\file{algorithms}{logical-agent-algorithm}}

The agent must be able to:\al
Represent states, actions, etc.\al
Incorporate new percepts\al
Update internal representations of the world\al
Deduce hidden properties of the world\al
Deduce appropriate actions




%%%%%%%%%%%% Slide %%%%%%%%%%%%%%%%%%%%%%%%%%%%%%%%%%%%%%%%%%%%%%%%%%%
\heading{Wumpus World PEAS description}

\begin{tabular}{lr}
\hbox{\begin{minipage}[b]{0.65\textwidth}
\note{Performance measure} \al
  gold +1000, death -1000\al
  -1 per step, -10 for using the arrow

\note{Environment}\al
Squares adjacent to wumpus are smelly\al
Squares adjacent to pit are breezy\al
Glitter iff gold is in the same square\al
Shooting kills wumpus if you are facing it\al
Shooting uses up the only arrow\al
Grabbing picks up gold if in same square\al
Releasing drops the gold in same square
\end{minipage}}
&
\epsfxsize=0.34\textwidth
\epsffile{\file{figures}{wumpus-world.ps}}
\end{tabular}

\note{Actuators} Left turn, Right turn,\nl
    Forward, Grab, Release, Shoot

\note{Sensors} Breeze, Glitter, Smell



%%%%%%%%%%%% Slide %%%%%%%%%%%%%%%%%%%%%%%%%%%%%%%%%%%%%%%%%%%%%%%%%%%
\heading{Wumpus world characterization}

\q{Observable} 

%%%%%%%%%%%% Slide %%%%%%%%%%%%%%%%%%%%%%%%%%%%%%%%%%%%%%%%%%%%%%%%%%%
\heading{Wumpus world characterization}

\q{Observable} No---only \note{local} perception

\q{Deterministic} 

%%%%%%%%%%%% Slide %%%%%%%%%%%%%%%%%%%%%%%%%%%%%%%%%%%%%%%%%%%%%%%%%%%
\heading{Wumpus world characterization}

\q{Observable} No---only \note{local} perception

\q{Deterministic} Yes---outcomes exactly specified

\q{Episodic} 

%%%%%%%%%%%% Slide %%%%%%%%%%%%%%%%%%%%%%%%%%%%%%%%%%%%%%%%%%%%%%%%%%%
\heading{Wumpus world characterization}

\q{Observable} No---only \note{local} perception

\q{Deterministic} Yes---outcomes exactly specified

\q{Episodic} No---sequential at the level of actions

\q{Static}  

%%%%%%%%%%%% Slide %%%%%%%%%%%%%%%%%%%%%%%%%%%%%%%%%%%%%%%%%%%%%%%%%%%
\heading{Wumpus world characterization}

\q{Observable} No---only \note{local} perception

\q{Deterministic} Yes---outcomes exactly specified

\q{Episodic} No---sequential at the level of actions

\q{Static}  Yes---Wumpus and Pits do not move

\q{Discrete} 

%%%%%%%%%%%% Slide %%%%%%%%%%%%%%%%%%%%%%%%%%%%%%%%%%%%%%%%%%%%%%%%%%%
\heading{Wumpus world characterization}

\q{Observable} No---only \note{local} perception

\q{Deterministic} Yes---outcomes exactly specified

\q{Episodic} No---sequential at the level of actions

\q{Static}  Yes---Wumpus and Pits do not move

\q{Discrete} Yes

\q{Single-agent} 

%%%%%%%%%%%% Slide %%%%%%%%%%%%%%%%%%%%%%%%%%%%%%%%%%%%%%%%%%%%%%%%%%%
\heading{Wumpus world characterization}

\q{Observable} No---only \note{local} perception

\q{Deterministic} Yes---outcomes exactly specified

\q{Episodic} No---sequential at the level of actions

\q{Static}  Yes---Wumpus and Pits do not move

\q{Discrete} Yes

\q{Single-agent} Yes---Wumpus is essentially a natural feature




%%%%%%%%%%%% Slide %%%%%%%%%%%%%%%%%%%%%%%%%%%%%%%%%%%%%%%%%%%%%%%%%%%
\heading{Exploring a wumpus world}

\vspace*{0.4in}

\epsfxsize=0.6\textwidth
\fig{\sfile{figures}{wumpus-seq0.ps}}

%%%%%%%%%%%% Slide %%%%%%%%%%%%%%%%%%%%%%%%%%%%%%%%%%%%%%%%%%%%%%%%%%%
\heading{Exploring a wumpus world}

\vspace*{0.4in}

\epsfxsize=0.6\textwidth
\fig{\sfile{figures}{wumpus-seq1.ps}}

%%%%%%%%%%%% Slide %%%%%%%%%%%%%%%%%%%%%%%%%%%%%%%%%%%%%%%%%%%%%%%%%%%
\heading{Exploring a wumpus world}

\vspace*{0.4in}

\epsfxsize=0.6\textwidth
\fig{\sfile{figures}{wumpus-seq2.ps}}

%%%%%%%%%%%% Slide %%%%%%%%%%%%%%%%%%%%%%%%%%%%%%%%%%%%%%%%%%%%%%%%%%%
\heading{Exploring a wumpus world}

\vspace*{0.4in}

\epsfxsize=0.6\textwidth
\fig{\sfile{figures}{wumpus-seq3.ps}}

%%%%%%%%%%%% Slide %%%%%%%%%%%%%%%%%%%%%%%%%%%%%%%%%%%%%%%%%%%%%%%%%%%
\heading{Exploring a wumpus world}

\vspace*{0.4in}

\epsfxsize=0.6\textwidth
\fig{\sfile{figures}{wumpus-seq4.ps}}

%%%%%%%%%%%% Slide %%%%%%%%%%%%%%%%%%%%%%%%%%%%%%%%%%%%%%%%%%%%%%%%%%%
\heading{Exploring a wumpus world}

\vspace*{0.4in}

\epsfxsize=0.6\textwidth
\fig{\sfile{figures}{wumpus-seq5.ps}}

%%%%%%%%%%%% Slide %%%%%%%%%%%%%%%%%%%%%%%%%%%%%%%%%%%%%%%%%%%%%%%%%%%
\heading{Exploring a wumpus world}

\vspace*{0.4in}

\epsfxsize=0.6\textwidth
\fig{\sfile{figures}{wumpus-seq6.ps}}

%%%%%%%%%%%% Slide %%%%%%%%%%%%%%%%%%%%%%%%%%%%%%%%%%%%%%%%%%%%%%%%%%%
\heading{Exploring a wumpus world}

\vspace*{0.4in}

\epsfxsize=0.6\textwidth
\fig{\sfile{figures}{wumpus-seq7.ps}}



%%%%%%%%%%%% Slide %%%%%%%%%%%%%%%%%%%%%%%%%%%%%%%%%%%%%%%%%%%%%%%%%%%
\heading{Other tight spots}

\begin{tabular}{lr}
\epsfxsize=2.7in
\raisebox{-0.5in}[2.5in]{\epsffile{\file{figures}{wumpus-bb.ps}}}
&
\hbox{\begin{minipage}[b]{0.5\textwidth}
Breeze in (1,2) and (2,1)\al
$\implies$ no safe actions\\

Assuming pits uniformly distributed,\\
(2,2) has pit w/ prob 0.86, vs.~0.31

\end{minipage}}
\\
   & \\
\epsfxsize=1.8in
\epsffile{\file{figures}{wumpus-s.ps}}
&
\hbox{\begin{minipage}[b]{0.65\textwidth}
Smell in (1,1) \al
$\implies$ cannot move

Can use a strategy of \defn{coercion}:\al
  shoot straight ahead\al
  wumpus was there $\implies$ dead $\implies$ safe\al
  wumpus wasn't there $\implies$ safe\al

\end{minipage}}
\end{tabular}




%%%%%%%%%%%% Slide %%%%%%%%%%%%%%%%%%%%%%%%%%%%%%%%%%%%%%%%%%%%%%%%%%%
\heading{Logic in general}

\defn{Logics} are formal languages for representing information\al
   such that conclusions can be drawn

\defn{Syntax} defines the sentences in the language

\defn{Semantics} define the ``meaning'' of sentences;\al
   i.e., define \defn{truth} of a sentence in a world

E.g., the language of arithmetic

\mat{$x+2 \geq y$} is a sentence; \mat{$x2+y>{}$} is not a sentence

\mat{$x+2 \geq y$} is true iff the number \mat{$x+2$} is no less
than the number \mat{$y$}

\mat{$x+2 \geq y$} is true in a world where \mat{$x\eq 7,\ y\eq 1$}\\
\mat{$x+2 \geq y$} is false in a world where \mat{$x\eq 0,\ y\eq 6$}


%%%%%%%%%%%% Slide %%%%%%%%%%%%%%%%%%%%%%%%%%%%%%%%%%%%%%%%%%%%%%%%%%%
\heading{Entailment}

\defn{Entailment} means that one thing \emph{follows from} another:
\mat{\[KB \models \alpha\]}

Knowledge base \mat{$KB$} entails sentence \mat{$\alpha$}\nl
   if and only if\\
\mat{$\alpha$} is true in all worlds where \mat{$KB$} is true

E.g., the KB containing ``the Giants won'' and ``the Reds won''\\
entails ``Either the Giants won or the Reds won''

E.g., \mat{$x+y\eq 4$} entails  \mat{$4\eq x+y$}

Entailment is a relationship between sentences (i.e., \emph{syntax})\\
that is based on \emph{semantics}

Note: brains process \emph{syntax} (of some sort)

%%%%%%%%%%%% Slide %%%%%%%%%%%%%%%%%%%%%%%%%%%%%%%%%%%%%%%%%%%%%%%%%%%
\heading{Models}

Logicians typically think in terms of \defn{models}, which are formally\\
structured worlds with respect to which truth can be evaluated

We say \mat{$m$} \note{is a model of} a sentence \mat{$\alpha$}
if \mat{$\alpha$} is true in \mat{$m$}

\mat{$M(\alpha)$} is the set of all models of \mat{$\alpha$}

Then \mat{$KB \models \alpha$} if and only if \mat{$M(KB) \subseteq M(\alpha)$}

\begin{tabular}{lr}
\hbox{\begin{minipage}[b]{0.6\textwidth}
E.g. \mat{$KB$} = Giants won and Reds won\nl
     \mat{$\alpha$} = Giants won
\end{minipage}}
&
\epsfxsize=3.2in
\raisebox{-2in}[0in]{\epsffile{\file{figures}{model-inclusion.ps}}}
\end{tabular}

%%%%%%%%%%%% Slide %%%%%%%%%%%%%%%%%%%%%%%%%%%%%%%%%%%%%%%%%%%%%%%%%%%
\heading{Entailment in the wumpus world}

\begin{tabular}{lr}
\hbox{\begin{minipage}[b]{0.55\textwidth}
Situation after detecting nothing in [1,1],\\
moving right, breeze in [2,1]\\
\\

Consider possible models for ?s\\
assuming only pits 
\end{minipage}}
&
\epsfxsize=0.4\textwidth
\epsffile{\file{figures}{wumpus-seq1c-alt.ps}}
\end{tabular}

3 Boolean choices $\implies$ 8 possible models


%%%%%%%%%%%% Slide %%%%%%%%%%%%%%%%%%%%%%%%%%%%%%%%%%%%%%%%%%%%%%%%%%%
\heading{Wumpus models}

\epsfxsize=0.75\textwidth
\fig{\file{figures}{wumpus-models1.ps}}

%%%%%%%%%%%% Slide %%%%%%%%%%%%%%%%%%%%%%%%%%%%%%%%%%%%%%%%%%%%%%%%%%%
\heading{Wumpus models}

\epsfxsize=0.75\textwidth
\fig{\file{figures}{wumpus-models2.ps}}

\mat{$KB$} = wumpus-world rules + observations

%%%%%%%%%%%% Slide %%%%%%%%%%%%%%%%%%%%%%%%%%%%%%%%%%%%%%%%%%%%%%%%%%%
\heading{Wumpus models}

\epsfxsize=0.75\textwidth
\fig{\file{figures}{wumpus-models3.ps}}

\mat{$KB$} = wumpus-world rules + observations

\mat{$\alpha_1$} = ``[1,2] is safe'', \mat{$KB \models \alpha_1$}, proved by \defn{model checking}


%%%%%%%%%%%% Slide %%%%%%%%%%%%%%%%%%%%%%%%%%%%%%%%%%%%%%%%%%%%%%%%%%%
\heading{Wumpus models}

\epsfxsize=0.75\textwidth
\fig{\file{figures}{wumpus-models2.ps}}

\mat{$KB$} = wumpus-world rules + observations

%%%%%%%%%%%% Slide %%%%%%%%%%%%%%%%%%%%%%%%%%%%%%%%%%%%%%%%%%%%%%%%%%%
\heading{Wumpus models}

\epsfxsize=0.75\textwidth
\fig{\file{figures}{wumpus-models4.ps}}

\mat{$KB$} = wumpus-world rules + observations

\mat{$\alpha_2$} = ``[2,2] is safe'', \mat{$KB  \not\models \alpha_2$}



%%%%%%%%%%%% Slide %%%%%%%%%%%%%%%%%%%%%%%%%%%%%%%%%%%%%%%%%%%%%%%%%%%
\heading{Inference}

\mat{$KB\vdash_i\alpha$} = sentence \mat{$\alpha$} can be derived from \mat{$KB$} by procedure \mat{$i$}

Consequences of \mat{$KB$} are a haystack; \mat{$\alpha$} is a needle.\\
Entailment = needle in haystack; inference = finding it

\defn{Soundness}: \mat{$i$} is sound if\nl
whenever \mat{$KB\vdash_i\alpha$}, it is also true that \mat{$KB\models\alpha$}

\defn{Completeness}: \mat{$i$} is complete if\nl
whenever \mat{$KB\models\alpha$}, it is also true that \mat{$KB\vdash_i\alpha$}

Preview: we will define a logic (first-order logic) which is
expressive enough to say almost anything of interest, and for which
there exists a sound and complete inference procedure.

That is, the procedure will answer any question whose answer follows
from what is known by the \mat{$KB$}.



%%%%%%%%%%%% Slide %%%%%%%%%%%%%%%%%%%%%%%%%%%%%%%%%%%%%%%%%%%%%%%%%%%
\heading{Propositional logic: Syntax}

Propositional logic is the simplest logic---illustrates basic ideas

The proposition symbols \mat{$P_1$}, \mat{$P_2$} etc are sentences

If \mat{$S$} is a sentence, \mat{$\lnot S$} is a sentence (\defn{negation})

If \mat{$S_1$} and \mat{$S_2$} are sentences, \mat{$S_1 \land S_2$} is a sentence (\defn{conjunction})
 
If \mat{$S_1$} and \mat{$S_2$} are sentences, \mat{$S_1 \lor S_2$} is a sentence (\defn{disjunction})

If \mat{$S_1$} and \mat{$S_2$} are sentences, \mat{$S_1 \implies S_2$} is a sentence (\defn{implication})

If \mat{$S_1$} and \mat{$S_2$} are sentences, \mat{$S_1 \lequiv S_2$} is a sentence (\defn{biconditional})





%%%%%%%%%%%% Slide %%%%%%%%%%%%%%%%%%%%%%%%%%%%%%%%%%%%%%%%%%%%%%%%%%%
\heading{Propositional logic: Semantics}

Each model specifies true/false for each proposition symbol

\begin{tabular}{lccc}
E.g. & \mat{$P_{1,2}$} & \mat{$P_{2,2}$} & \mat{$P_{3,1}$}\\
     & \mat{$true$}   &  \mat{$true$} &  \mat{$false$}
\end{tabular}

(With these symbols, 8 possible models, can be enumerated automatically.)

Rules for evaluating truth with respect to a model \mat{$m$}:

\begin{tabular}{rcclcl}
\mat{$\lnot S$}          & is true iff & \mat{$S$} & is false & & \\
\mat{$S_1 \land S_2$}    & is true iff & \mat{$S_1$} & is true \emph{and} & \mat{$S_2$} & is true\\
\mat{$S_1 \lor S_2$}     & is true iff & \mat{$S_1$} & is true \emph{or} & \mat{$S_2$} & is true\\
\mat{$S_1 \implies S_2$} & is true iff& \mat{$S_1$} & is false \emph{or} & \mat{$S_2$} & is true\\
\qquad i.e.,  & is false iff& \mat{$S_1$} & is true \emph{and} & \mat{$S_2$} & is false\\
\mat{$S_1 \lequiv S_2$} & is true iff& \mat{$S_1\implies S_2$} & is true \emph{and} & \mat{$S_2\implies S_1$} & is true
\end{tabular}

Simple recursive process evaluates an arbitrary sentence, e.g.,\\
\mat{$\lnot P_{1,2}\land (P_{2,2}\lor P_{3,1})$} = \mat{$true\land (false \lor true)\eq true\land
true \eq true$}

%%%%%%%%%%%% Slide %%%%%%%%%%%%%%%%%%%%%%%%%%%%%%%%%%%%%%%%%%%%%%%%%%%
\heading{Truth tables for connectives}

\begin{center}
\input{\file{tables}{truth-tables-table}}
\end{center}

%%%%%%%%%%%% Slide %%%%%%%%%%%%%%%%%%%%%%%%%%%%%%%%%%%%%%%%%%%%%%%%%%%
\heading{Wumpus world sentences}

Let \mat{$P_{i,j}$} be true if there is a pit in \mat{$[i,j]$}.\\
Let \mat{$B_{i,j}$} be true if there is a breeze in \mat{$[i,j]$}.
\mat{\begin{eqnarray*}
 \lnot P_{1,1}\\
 \lnot B_{1,1}\\
 B_{2,1}
\end{eqnarray*}}
``Pits cause breezes in adjacent squares''

%%%%%%%%%%%% Slide %%%%%%%%%%%%%%%%%%%%%%%%%%%%%%%%%%%%%%%%%%%%%%%%%%%
\heading{Wumpus world sentences}

Let \mat{$P_{i,j}$} be true if there is a pit in \mat{$[i,j]$}.\\
Let \mat{$B_{i,j}$} be true if there is a breeze in \mat{$[i,j]$}.
\mat{\begin{eqnarray*}
 \lnot P_{1,1}\\
 \lnot B_{1,1}\\
 B_{2,1}
\end{eqnarray*}}
``Pits cause breezes in adjacent squares''
\mat{\begin{eqnarray*}
 B_{1,1} &\lequiv& (P_{1,2} \lor P_{2,1})\\
 B_{2,1} &\lequiv& (P_{1,1} \lor P_{2,2}\lor P_{3,1})
\end{eqnarray*}}
``A square is breezy \emph{if and only if} there is an adjacent pit''


%%%%%%%%%%%% Slide %%%%%%%%%%%%%%%%%%%%%%%%%%%%%%%%%%%%%%%%%%%%%%%%%%%
\heading{Truth tables for inference}

\begin{LARGE}
\begin{center}
\input{\file{tables}{wumpus-model-table}}
\end{center}
\end{LARGE}

Enumerate rows (different assignments to symbols),\\
if \mat{KB} is true in row, check that \mat{$\alpha$} is too


%%%%%%%%%%%% Slide %%%%%%%%%%%%%%%%%%%%%%%%%%%%%%%%%%%%%%%%%%%%%%%%%%%
\heading{Inference by enumeration}

Depth-first enumeration of all models is sound and complete 

\input{\file{algorithms}{tt-entails-algorithm}}

\mat{$O(2^n)$} for \mat{$n$} symbols; problem is \emph{co-NP-complete}

%%%%%%%%%%%% Slide %%%%%%%%%%%%%%%%%%%%%%%%%%%%%%%%%%%%%%%%%%%%%%%%%%%
\heading{Logical equivalence}

Two sentences are \defn{logically equivalent} iff true in same models:\al
\mat{$\alpha\equiv\beta$} if and only if \mat{$\alpha\models\beta$} 
and \mat{$\beta\models\alpha$}

\input{\file{tables}{logical-equivalence-table}}

%%%%%%%%%%%% Slide %%%%%%%%%%%%%%%%%%%%%%%%%%%%%%%%%%%%%%%%%%%%%%%%%%%
\heading{Validity and satisfiability}

A sentence is \defn{valid} if it is true in \emph{all} models,\nl
e.g., \mat{$True$},\quad \mat{$A \lor \lnot A$}, \quad \mat{$A \implies A$}, \quad 
      \mat{$(A \land (A \implies B)) \implies B$}

Validity is connected to inference via the \defn{Deduction Theorem}:\nl
      \mat{$KB \models \alpha$} if and only if \mat{$(KB \implies \alpha)$} is valid

A sentence is \defn{satisfiable} if it is true in \emph{some} model\nl
e.g., \mat{$A\lor B$},\qquad \mat{$C$}

A sentence is \defn{unsatisfiable} if it is true in \emph{no} models\nl
e.g., \mat{$A\land \lnot A$}

Satisfiability is connected to inference via the following:\nl
      \mat{$KB \models \alpha$} if and only if \mat{$(KB \land \lnot \alpha)$} is unsatisfiable\\
i.e., prove \mat{$\alpha$} by \note{{\it reductio ad absurdum}}


%%%%%%%%%%%% Slide %%%%%%%%%%%%%%%%%%%%%%%%%%%%%%%%%%%%%%%%%%%%%%%%%%%
\heading{Proof methods}

Proof methods divide into (roughly) two kinds:\al

\note{Application of inference rules}\al
    -- Legitimate (sound) generation of new sentences from old\al
    -- \defn{Proof} = a sequence of inference rule applications\nl
       Can use inference rules as operators in a standard search alg.\al
    -- Typically require translation of sentences into a \defn{normal form}

\note{Model checking}\al
    truth table enumeration (always exponential in \mat{$n$})\al
    improved backtracking, e.g., Davis--Putnam--Logemann--Loveland\al
    heuristic search in model space (sound but incomplete)\nl
       e.g., min-conflicts-like hill-climbing algorithms




%%%%%%%%%%%% Slide %%%%%%%%%%%%%%%%%%%%%%%%%%%%%%%%%%%%%%%%%%%%%%%%%%%
\heading{Forward and backward chaining}

\defn{Horn Form} (restricted)\nl
    KB = \emph{conjunction} of \emph{Horn clauses}\al
    Horn clause = \nl
     \blob proposition symbol;  or\nl
     \blob (conjunction of symbols) \mat{$\implies$} symbol\al
    E.g., \mat{$C \land (B \implies A) \land (C \land D \implies B)$}

\defn{Modus Ponens} (for Horn Form): complete for Horn KBs
\mat{\[\frac{\alpha_1,\ldots,\alpha_n,\qquad \alpha_1\land \cdots \land \alpha_n\implies \beta}{\beta} 
\]}
Can be used with \defn{forward chaining} or \defn{backward chaining}.\\
These algorithms are very natural and run in \emph{linear} time

%%%%%%%%%%%% Slide %%%%%%%%%%%%%%%%%%%%%%%%%%%%%%%%%%%%%%%%%%%%%%%%%%%
\heading{Forward chaining}

Idea: fire any rule whose premises are satisfied in the \mat{$KB$},\nl
   add its conclusion to the \mat{$KB$}, until query is found

\begin{minipage}{0.48\maxfigwidth}
\noindent 
\tab\tab\tab\tab\tab\tab$P\implies Q$\\[4pt]
\tab\tab\tab\tab\tab\tab$L\land M \implies P$\\[4pt]
\tab\tab\tab\tab\tab\tab$B \land L \implies M$\\[4pt]
\tab\tab\tab\tab\tab\tab$A \land P\implies L$\\[4pt]
\tab\tab\tab\tab\tab\tab$A \land B\implies L$\\[4pt]
\tab\tab\tab\tab\tab\tab$A$\\[4pt]
\tab\tab\tab\tab\tab\tab$B$

\end{minipage}\hfill
\begin{minipage}{0.48\maxfigwidth}
\epsfxsize=0.25\maxfigwidth
\centerline{\epsffile{\file{figures}{pl-horn-example.ps}}}
\end{minipage}

%%%%%%%%%%%% Slide %%%%%%%%%%%%%%%%%%%%%%%%%%%%%%%%%%%%%%%%%%%%%%%%%%%
\heading{Forward chaining algorithm}

\input{\file{algorithms}{pl-fc-algorithm}}

%%%%%%%%%%%% Slide %%%%%%%%%%%%%%%%%%%%%%%%%%%%%%%%%%%%%%%%%%%%%%%%%%%
\heading{Forward chaining example}

\epsfxsize=0.45\maxfigwidth
\fig{\file{figures}{fc-horn-example01c.ps}}

%%%%%%%%%%%% Slide %%%%%%%%%%%%%%%%%%%%%%%%%%%%%%%%%%%%%%%%%%%%%%%%%%%
\heading{Forward chaining example}

\epsfxsize=0.45\maxfigwidth
\fig{\file{figures}{fc-horn-example02c.ps}}

%%%%%%%%%%%% Slide %%%%%%%%%%%%%%%%%%%%%%%%%%%%%%%%%%%%%%%%%%%%%%%%%%%
\heading{Forward chaining example}

\epsfxsize=0.45\maxfigwidth
\fig{\file{figures}{fc-horn-example03c.ps}}

%%%%%%%%%%%% Slide %%%%%%%%%%%%%%%%%%%%%%%%%%%%%%%%%%%%%%%%%%%%%%%%%%%
\heading{Forward chaining example}

\epsfxsize=0.45\maxfigwidth
\fig{\file{figures}{fc-horn-example04c.ps}}

%%%%%%%%%%%% Slide %%%%%%%%%%%%%%%%%%%%%%%%%%%%%%%%%%%%%%%%%%%%%%%%%%%
\heading{Forward chaining example}

\epsfxsize=0.45\maxfigwidth
\fig{\file{figures}{fc-horn-example05c.ps}}

%%%%%%%%%%%% Slide %%%%%%%%%%%%%%%%%%%%%%%%%%%%%%%%%%%%%%%%%%%%%%%%%%%
\heading{Forward chaining example}

\epsfxsize=0.45\maxfigwidth
\fig{\file{figures}{fc-horn-example06c.ps}}

%%%%%%%%%%%% Slide %%%%%%%%%%%%%%%%%%%%%%%%%%%%%%%%%%%%%%%%%%%%%%%%%%%
\heading{Forward chaining example}

\epsfxsize=0.45\maxfigwidth
\fig{\file{figures}{fc-horn-example07c.ps}}

%%%%%%%%%%%% Slide %%%%%%%%%%%%%%%%%%%%%%%%%%%%%%%%%%%%%%%%%%%%%%%%%%%
\heading{Forward chaining example}

\epsfxsize=0.45\maxfigwidth
\fig{\file{figures}{fc-horn-example08c.ps}}

%%%%%%%%%%%% Slide %%%%%%%%%%%%%%%%%%%%%%%%%%%%%%%%%%%%%%%%%%%%%%%%%%%
\heading{Proof of completeness}

FC derives every atomic sentence that is entailed by \mat{$KB$}

1. FC reaches a \defn{fixed point} where no new atomic sentences are derived

2. Consider the final state as a model \mat{$m$}, assigning true/false to symbols

3. Every clause in the original \mat{$KB$} is true in \mat{$m$}\al
   \emph{Proof}:  Suppose a clause \mat{$a_1\land\ldots\land a_k\textimplies b$} is false in \mat{$m$}\al
   Then \mat{$a_1\land\ldots\land a_k$} is true in \mat{$m$} and \mat{$b$} is false in \mat{$m$}\al
   Therefore the algorithm has not reached a fixed point!

4. Hence \mat{$m$} is a model of \mat{$KB$}

5. If \mat{$KB\models q$}, \mat{$q$} is true in \emph{every} model of \mat{$KB$}, including \mat{$m$}

\note{General idea}: construct any model of \mat{$KB$} by sound inference, check \mat{$\alpha$}

%%%%%%%%%%%% Slide %%%%%%%%%%%%%%%%%%%%%%%%%%%%%%%%%%%%%%%%%%%%%%%%%%%
\heading{Backward chaining}

Idea: work backwards from the query \mat{$q$}:\al
   to prove \mat{$q$} by BC,\nl
      check if \mat{$q$} is known already, or\nl
      prove by BC all premises of some rule concluding \mat{$q$}

Avoid loops: check if new subgoal is already on the goal stack

Avoid repeated work: check if new subgoal \al
  1) has already been proved true, or\al
  2) has already failed

%%%%%%%%%%%% Slide %%%%%%%%%%%%%%%%%%%%%%%%%%%%%%%%%%%%%%%%%%%%%%%%%%%
\heading{Backward chaining example}

\epsfxsize=0.45\maxfigwidth
\fig{\file{figures}{bc-horn-example01c.ps}}

%%%%%%%%%%%% Slide %%%%%%%%%%%%%%%%%%%%%%%%%%%%%%%%%%%%%%%%%%%%%%%%%%%
\heading{Backward chaining example}

\epsfxsize=0.45\maxfigwidth
\fig{\file{figures}{bc-horn-example02c.ps}}

%%%%%%%%%%%% Slide %%%%%%%%%%%%%%%%%%%%%%%%%%%%%%%%%%%%%%%%%%%%%%%%%%%
\heading{Backward chaining example}

\epsfxsize=0.45\maxfigwidth
\fig{\file{figures}{bc-horn-example03c.ps}}

%%%%%%%%%%%% Slide %%%%%%%%%%%%%%%%%%%%%%%%%%%%%%%%%%%%%%%%%%%%%%%%%%%
\heading{Backward chaining example}

\epsfxsize=0.45\maxfigwidth
\fig{\file{figures}{bc-horn-example04c.ps}}

%%%%%%%%%%%% Slide %%%%%%%%%%%%%%%%%%%%%%%%%%%%%%%%%%%%%%%%%%%%%%%%%%%
\heading{Backward chaining example}

\epsfxsize=0.45\maxfigwidth
\fig{\file{figures}{bc-horn-example03c.ps}}

%%%%%%%%%%%% Slide %%%%%%%%%%%%%%%%%%%%%%%%%%%%%%%%%%%%%%%%%%%%%%%%%%%
\heading{Backward chaining example}

\epsfxsize=0.45\maxfigwidth
\fig{\file{figures}{bc-horn-example05c.ps}}

%%%%%%%%%%%% Slide %%%%%%%%%%%%%%%%%%%%%%%%%%%%%%%%%%%%%%%%%%%%%%%%%%%
\heading{Backward chaining example}

\epsfxsize=0.45\maxfigwidth
\fig{\file{figures}{bc-horn-example06c.ps}}

%%%%%%%%%%%% Slide %%%%%%%%%%%%%%%%%%%%%%%%%%%%%%%%%%%%%%%%%%%%%%%%%%%
\heading{Backward chaining example}

\epsfxsize=0.45\maxfigwidth
\fig{\file{figures}{bc-horn-example07c.ps}}

%%%%%%%%%%%% Slide %%%%%%%%%%%%%%%%%%%%%%%%%%%%%%%%%%%%%%%%%%%%%%%%%%%
\heading{Backward chaining example}

\epsfxsize=0.45\maxfigwidth
\fig{\file{figures}{bc-horn-example08c.ps}}

%%%%%%%%%%%% Slide %%%%%%%%%%%%%%%%%%%%%%%%%%%%%%%%%%%%%%%%%%%%%%%%%%%
\heading{Backward chaining example}

\epsfxsize=0.45\maxfigwidth
\fig{\file{figures}{bc-horn-example09c.ps}}

%%%%%%%%%%%% Slide %%%%%%%%%%%%%%%%%%%%%%%%%%%%%%%%%%%%%%%%%%%%%%%%%%%
\heading{Backward chaining example}

\epsfxsize=0.45\maxfigwidth
\fig{\file{figures}{bc-horn-example10c.ps}}


%%%%%%%%%%%% Slide %%%%%%%%%%%%%%%%%%%%%%%%%%%%%%%%%%%%%%%%%%%%%%%%%%%
\heading{Forward vs.~backward chaining}

FC is \defn{data-driven}, cf. automatic, unconscious processing,\nl
e.g., object recognition, routine decisions

May do lots of work that is irrelevant to the goal

BC is \defn{goal-driven}, appropriate for problem-solving,\nl
e.g., Where are my keys? How do I get into a PhD program?

Complexity of BC can be \emph{much less} than linear in size of KB

%%%%%%%%%%%% Slide %%%%%%%%%%%%%%%%%%%%%%%%%%%%%%%%%%%%%%%%%%%%%%%%%%%
\heading{Resolution}

\defn{Conjunctive Normal Form} (CNF---universal)\nl
    \emph{conjunction} of $\underbrace{\mbox{\emph{disjunctions} of \emph{literals}}}$\nl
    \phantom{\emph{conjunction} of \emph{disjuncti}}\emph{clauses}\nl
    E.g., \mat{$(A \lor \lnot B) \land (B \lor \lnot C \lor \lnot D)$}

\defn{Resolution} inference rule (for CNF): complete for propositional logic
\mat{\[\frac {\ell_1 \lor \cdots\lor \ell_k,\qquad m_1 \lor \cdots\lor m_n}
        {\ell_1 \lor \cdots\lor \ell_{i-1}\lor \ell_{i+1}\lor\cdots\lor \ell_k
        \lor m_1 \lor \cdots \lor m_{j-1}\lor m_{j+1}\lor\cdots\lor m_n}
\]}
where \mat{$\ell_i$} and \mat{$m_j$} are complementary literals. E.g.,\hfill\epsfxsize=1.75in\raisebox{-1.5in}[0pt][0pt]{\epsffile{\file{figures}{wumpus-seq5c.ps}}}
\mat{\[
   \frac{P_{1,3} \lor P_{2,2},\qquad\lnot P_{2,2}}
       {P_{1,3}}
\]}
Resolution is sound and complete for propositional logic

%%%%%%%%%%%% Slide %%%%%%%%%%%%%%%%%%%%%%%%%%%%%%%%%%%%%%%%%%%%%%%%%%%
\heading{Conversion to CNF}

\mat{$B_{1,1} \textlequiv (P_{1,2} \lor P_{2,1})$}

1.  Eliminate \mat{$\lequivsymbol$}, replacing \mat{$\alpha\textlequiv \beta$} with \mat{$(\alpha\implies
\beta)\land (\beta\implies \alpha)$}.
\mat{\[
   (B_{1,1} \implies (P_{1,2} \lor P_{2,1})) \land 
   ((P_{1,2} \lor P_{2,1})\implies B_{1,1})
\]}
2. Eliminate \mat{$\impliessymbol$}, replacing \mat{$\alpha\textimplies \beta$} with \mat{$\lnot
\alpha\lor \beta$}.
\mat{\[
   (\lnot B_{1,1} \lor P_{1,2} \lor P_{2,1}) \land 
   (\lnot(P_{1,2} \lor P_{2,1})\lor B_{1,1})
\]}
3. Move \mat{$\lnot$} inwards using de Morgan's rules and double-negation:
\mat{\[
   (\lnot B_{1,1} \lor P_{1,2} \lor P_{2,1}) \land 
   ((\lnot P_{1,2} \land \lnot P_{2,1})\lor B_{1,1})
\]}
4. Apply distributivity law (\mat{${\lor}$} over \mat{${\land}$}) and flatten:
\mat{\[
   (\lnot B_{1,1} \lor P_{1,2} \lor P_{2,1}) \land 
   (\lnot P_{1,2} \lor B_{1,1}) \land (\lnot P_{2,1} \lor B_{1,1})
\]}

%%%%%%%%%%%% Slide %%%%%%%%%%%%%%%%%%%%%%%%%%%%%%%%%%%%%%%%%%%%%%%%%%%
\heading{Resolution algorithm}

Proof by contradiction, i.e., show \mat{$KB\land\lnot\alpha$} unsatisfiable

\input{\file{algorithms}{pl-resolution-algorithm}}

%%%%%%%%%%%% Slide %%%%%%%%%%%%%%%%%%%%%%%%%%%%%%%%%%%%%%%%%%%%%%%%%%%
\heading{Resolution example}

\mat{$KB = (B_{1,1} \lequiv (P_{1,2} \lor P_{2,1})) \land \lnot B_{1,1}$}
\mat{$\alpha = \lnot P_{1,2}$}

\vspace*{0.5in}

\epsfxsize=1.1\textwidth
\fig{\file{figures}{wumpus-resolution.ps}}


%%%%%%%%%%%% Slide %%%%%%%%%%%%%%%%%%%%%%%%%%%%%%%%%%%%%%%%%%%%%%%%%%%
\heading{Summary}

Logical agents apply \defn{inference} to a \defn{knowledge base}\al
to derive new information and make decisions

Basic concepts of logic:\al
-- \defn{syntax}: formal structure of \defn{sentences}\al
-- \defn{semantics}: \defn{truth} of sentences wrt \defn{models}\al
-- \defn{entailment}: necessary truth of one sentence given another\al
-- \defn{inference}: deriving sentences from other sentences\al
-- \defn{soundess}: derivations produce only entailed sentences\al
-- \defn{completeness}: derivations can produce all entailed sentences

Wumpus world requires the ability to represent partial
and negated information, reason by cases, etc.

Forward, backward chaining are linear-time, complete for Horn clauses\\
Resolution is complete for propositional logic

Propositional logic lacks expressive power


\end{huge} 
\end{document}
